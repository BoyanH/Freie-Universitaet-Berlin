\input{src/header}											% bindet Header ein (WICHTIG)
\usepackage{graphicx}
\usepackage{fancyvrb}

\newcommand{\dozent}{Prof. Dr. Agn`es Voisard, Nicolas Lehmann}					% <-- Names des Dozenten eintragen
\newcommand{\tutor}{Nicolas Lehmann}						% <-- Name eurer Tutoriun eintragen
\newcommand{\tutoriumNo}{10}				% <-- Nummer im KVV nachschauen
\newcommand{\projectNo}{4}									% <-- Nummer des Übungszettels
\newcommand{\veranstaltung}{Datenbanksysteme}	% <-- Name der Lehrveranstaltung eintragen
\newcommand{\semester}{SoSe 2017}						% <-- z.B. SoSe 17, WiSe 17/18
\newcommand{\studenten}{Boyan Hristov, Julian Habib}			% <-- Hier eure Namen eintragen
% /////////////////////// BEGIN DOKUMENT /////////////////////////


\begin{document}
% /////////////////////// BEGIN TITLEPAGE /////////////////////////
\begin{titlepage}
	\subject{\dozent}
	\title{\veranstaltung, \semester}
	\subtitle{\Large Übungsblatt \projectNo\\ \large\vspace{1ex} }
	\author{\studenten}
	\date{\normalsize \today}
\end{titlepage}

\maketitle								% Erstellt das Titelblatt
\vspace*{-9cm}							% rückt Logo an den oberen Seitenrand
\makebox[\dimexpr\textwidth+1cm][r]{	%rechtsbündig und geht rechts 1cm über Layout hinaus
	\includegraphics[width=0.4\textwidth]{src/fu_logo} % fügt FU-Logo ein
}
% /////////////////////// END TITLEPAGE /////////////////////////

\vspace{7cm}							% Abstand
\rule{\linewidth}{0.8pt}				% horizontale Linie										% erstellt die Titelseite


Link zum Git Repository: \url{https://github.com/BoyanH/Freie-Universitaet-Berlin/tree/master/Datenbanksysteme/Solutions/homework4}

% /////////////////////// Aufgabe 1 /////////////////////////
\section{Aufgabe}


\begin{itemize}
	\item[a)]

	\begin{lstlisting}[style=Bash]
	[hristov@Edgy ~]$ sudo su // als Root Nutzer sich einmelden (Root rechte für die folgende Operationen notwendig, man kann auch jedes mal sudo schreiben)
	[root@Edgy hristov]# systemctl start postgresql.service // starte den service für postgresql, damit wir postgresql benutzen können
	[root@Edgy hristov]# su -l postgres // zu postgres nutzer wächseln 
	[postgres@Edgy ~]$ exit // wieder zu root Nutzer wechseln, da postgres keine Root Rechte hat und wir den Password dafür nicht mehr kennen (oops)
	[root@Edgy hristov]# passwd postgres -d // lösche den Password für Nutzer postgres
	[root@Edgy hristov]# passwd postgres // verändere den password. Zwei Console Prompts folgen
	New password: postgres
	Retype new password: postgres
	[root@Edgy hristov]# su -l postgres
	[postgres@Edgy ~]$ createdb testdb // erstelle Datenbank 'testdb'
	[postgres@Edgy ~]$ dropdb testdb // lösche Datenbank 'testdb'
	[postgres@Edgy ~]$ createdb dbs // erstelle Datenbank 'testdb'	
	\end{lstlisting}

	\item[b)]

	\begin{lstlisting}[style=Bash]
		[hristov@Edgy homework4]$ sudo su -l postgres // als postgres Nutzer sich anmelden
		[postgres@Edgy ~]$ createuser testuser --password --interactive
		Shall the new role be a superuser? (y/n) n
		Shall the new role be allowed to create databases? (y/n) y
		Shall the new role be allowed to create more new roles? (y/n) y
		Password: testpassword // Erstelle Nutzer 'testuser' mit Password 'testpassword', der Datenbanken und Rollen erstellen kann, aber keine Superrechte hat	
		[postgres@Edgy ~]$ psql // in PostgreSQL Shell wächseln
		postgres=# ALTER USER "testuser" WITH PASSWORD 'testpass'; // Password von 'testuser' ändern
		postgres=# \q
		[postgres@Edgy ~]$ exit
		logout
		[hristov@Edgy homework4]$ // wieder zu eigenen Nutzer wächseln 

	\end{lstlisting}

	\item[c)]

	\begin{lstlisting}[style=Bash]
		[hristov@Edgy homework4]$ sudo su -l postgres
		[postgres@Edgy ~]$ psql -d dbs -U testuser
		dbs=> CREATE TABLE Student ( // erstelle Tabelle wie in der Übungsaufgabe beschrieben
		matrikelnummer integer NOT NULL,
		vorname character varying(20) NOT NULL,
		nachname character varying(20) NOT NULL     
		);                                                     
		CREATE TABLE // output
		dbs=> \dt
		          List of relations
		 Schema |  Name   | Type  |  Owner   
		--------+---------+-------+----------
		 public | student | table | testuser
		(1 row)

	\end{lstlisting}

	\item[d)]

	\begin{lstlisting}[style=Bash]
		dbs=> ALTER TABLE Student
		ADD PRIMARY KEY (matrikelnummer)
		;
		ALTER TABLE
	\end{lstlisting}

\end{itemize}

\section{Aufgabe}

\begin{itemize}

\item[a)]
\begin{Verbatim}
SELECT Vorname, Nachname
FROM Passagier
WHERE Alter > 42
\end{Verbatim}

\item[b)]
\begin{Verbatim}
SELECT P.Kreditkarennummer
FROM Passagier P, Fluggesellschaft G, Flug F
WHERE P.ID = F.Passagier-ID AND
	F.Fluggesellschaft-ID = G.ID AND
	Datum > 18.01.2014 AND Datum < 27.09.2014
\end{Verbatim}

\item[c)]
\begin{Verbatim}
SELECT G.Name
FROM Flug F, Wetter W, Fluggesellschaft G 
WHERE F.Datum = W.Datum AND
	F.Fluggesellschaft-ID = G.ID AND
	W.Sonnenscheindauer > 8
\end{Verbatim}

\item[d)]
\begin{Verbatim}
SELECT P.Vorname, P.Nachname
FROM Passagier P, Flug F, Wetter W 
WHERE F.Passagier-ID = P.ID AND
	F.Datum = W.Datum AND
	NOT (Temperatur > 20 AND Regenmenge < 10 AND Sonnenscheindauer > 6)
\end{Verbatim}

\end{itemize}

\section{Aufgabe}

\begin{itemize}

\item[а)]
\begin{Verbatim}
SELECT DISTINCT TOP 3 Name
FROM (
	SELECT F.Fluggesellschaft-ID, F.Datum, Count(*), G.Name
	FROM FLuggesellschaft G, Flug F
	WHERE G.ID = F.Fluggesellschaft-ID
	GROUP BY F.Fluggesellschaft-ID, F.Datum
	ORDER BY Count(*)
	)
\end{Verbatim}

\item[b)]
\begin{Verbatim}
SELECT DISTINCT TOP 10 Vorname, Nachname
FROM (
	SELECT P.Vorname, P.Nachname, F.Passagier-ID, Count(*)
	FROM Passagier P, Flug F
	WHERE P.ID = F.Passagier-ID
	GROUP BY F.Passagier-ID
	ORDER BY Count(*) ASC
	)
WHERE
\end{Verbatim}

\item[c)]

Da am meisten und am wenigsten mit verschieden zwei separate Aussagen sind gibt es zwei Möglichkeiten die Aufgabe zu bearbeiten. Entweder zu sehen wie oft jedes Passagier mit je Fluggesellschaft geflogen ist und dann nach den Flüge die Resultate zu ordnen, oder als 2 separate stabile Sortierungen - erstmal nach Flüge absteigend, danach nach verschiedene Fluggesellschaften aufsteigend und am Ende die Top 5 selektieren. Wir haben uns für die 2. Variante geeinigt (klingt sinnvoller). Wir gehen hier davon aus, dass ORDER BY in SQL stabil ist.

\begin{Verbatim}
SELECT DISTINCT TOP 5 Vorname, Nachname
FROM (
	SELECT Fluggesellschaft-ID, Count(*), Vorname, Nachname
	FROM (
		SELECT P.Vorname, P.Nachname, F.Passagier-ID, F.Fluggesellschaft-ID
		FROM Passagier P, Flug F
		WHERE P.ID = F.Passagier-ID
		GROUP BY F.Passagier-ID
		ORDER BY Count(*) DESC
		)
	WHERE
	GROUP BY Fluggesellschaft-ID
	ORDER BY Count(*) ASC
	)
WHERE
\end{Verbatim}


\end{itemize}

\section{Aufgabe}

\begin{itemize}

\item[a)]

Den cabal-install tool installieren
\begin{lstlisting}[style=Bash]
	[hristov@Edgy exercise4]$ sudo pacman -S cabal-install
\end{lstlisting}
Letzten cabal Package List von hackage.haskell.org herunterladen
\begin{lstlisting}[style=Bash]
	[hristov@Edgy exercise4]$ cabal update
\end{lstlisting}
Und postgresql-simple Haskell Module herunterladen
\begin{lstlisting}[style=Bash]
	[hristov@Edgy exercise4]$ cabal install postgresql-simple
\end{lstlisting}

\begin{lstlisting}[style=hs]
{-# LANGUAGE OverloadedStrings #-}

import Control.Monad
import Control.Applicative
import Database.PostgreSQL.Simple
import qualified Data.Text as Text


main = do
	conn <- connectPostgreSQL "dbname='dbs' user='testuser' host='localhost' password='testpass'"
	putStrLn "Connected to DB! Querying all students"
	xs <- (query_ conn "SELECT vorname, nachname, matrikelnummer FROM Student")
	forM_ xs $ \(fname, lname, matrikelnummer) ->
  		putStrLn $ Text.unpack "Matrikelnummer: " ++ show (matrikelnummer :: Int) ++ "; Vorname: " ++ fname ++ "; Nachname: " ++ lname
\end{lstlisting}


\item[b)]

Erstmal die Bibliothek für Arbeit mit PostgreSQL installieren.

\begin{lstlisting}[style=Bash]
	[hristov@Edgy exercise4]$ sudo pacman -S python-psycopg2
\end{lstlisting}

PostgreSQL Service starten

\begin{lstlisting}[style=Bash]
	[hristov@Edgy exercise4]$ sudo systemctl start postgresql.service
\end{lstlisting}

\begin{lstlisting}[style=py]
	import psycopg2
import uuid

class Student:
	
	def __init__(self, fname, lname):
		self.firstName = fname
		self.lastName = lname
		self.id = uuid.uuid4().int & (1<<16)-1 # turn into a 16 bit integer with mask

	def addToDB(self, connection):
		c = connection.cursor()

		c.execute("SELECT * FROM Student")
		students = c.fetchall()
		print("Students in Student table: ")
		for student in students:
			print(student)

		insertStudentSql = "INSERT INTO Student (vorname, nachname, matrikelnummer) VALUES ('{0}', '{1}', '{2}')"
		sqlCommand = insertStudentSql.format(self.firstName, self.lastName, self.id)
		c.execute(sqlCommand)
		connection.commit()

dbUser = "testuser"
dbName = "dbs"
host = "localhost"
dbUserPassword = "testpass"

connectionStringTemplate = "dbname='{0}' user='{1}' host='{2}' password='{3}'"
connectionString = connectionStringTemplate.format(dbName, dbUser, host, dbUserPassword)

try:
	connection = psycopg2.connect(connectionString)
	print("Connected to DB!")
	randomStudent = Student("Will", "Smith")
	randomStudent.addToDB(connection)
	print("Successfully addded Student to DB!")
	   
	connection.close()   
except Exception as er:
	print("Failed connecting to DB!")
	print(er)
\end{lstlisting}

\item[c)]

Erstmal den Postgresql JDBC Driver für Java herunterladen

\begin{lstlisting}[style=Bash]
	[hristov@Edgy ~]$ yaourt postgresql-jdbc
\end{lstlisting}

Füge das .jar zu den Classpath von dem DBTest.java

z.B javac -classpath postgresql-42.1.1.jre6.jar DBTetst.java

\begin{lstlisting}[style=java]
package fu.alp4;

import java.sql.Connection;
import java.sql.DriverManager;
import java.sql.ResultSet;
import java.sql.SQLException;
import java.sql.Statement;
import java.util.logging.Level;
import java.util.logging.Logger;


public class DBTest {

    public static void main(String[] args) {

        try {

            Class.forName("org.postgresql.Driver");

        } catch (ClassNotFoundException e) {

            System.out.println("PostgreSQL JDBC Driver not included in class path!");
            return;

        }

        System.out.println("JDBC registered succesffully :)");

        Connection connection = null;
        Statement statement = null;
        ResultSet rs = null;

        try {
            connection = DriverManager.getConnection("jdbc:postgresql://localhost/dbs", "testuser", "testpass");
            statement = connection.createStatement();
        } catch (SQLException e) {

            System.out.println("Failed connecting to DB!");
            return;

        }

        System.out.println("Successfully connected to DB!");


        try {
            System.out.println("Students in DB: ");
            rs = statement.executeQuery("SELECT matrikelnummer, vorname, nachname FROM Student");
            while (rs.next()) {
                int matrikelnummer = rs.getInt("matrikelnummer");
                String firstName = rs.getString("vorname");
                String lastName = rs.getString("nachname");
                System.out.println(String.format("Matrikelnummer: %s; Vorname: %s; Nachname: %s",
                        matrikelnummer, firstName, lastName));
            }
            connection.close();
        }
        catch(SQLException e) {
            System.out.println("Error while executing sql query!");
        }

    }
}

\end{lstlisting}

\end{itemize}

% /////////////////////// END DOKUMENT /////////////////////////
\end{document}
