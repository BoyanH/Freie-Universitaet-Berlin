\input{src/header}											% bindet Header ein (WICHTIG)
\usepackage{graphicx}
\usepackage{fancyvrb}

\newcommand{\dozent}{Prof. Dr. Agn`es Voisard, Nicolas Lehmann}					% <-- Names des Dozenten eintragen
\newcommand{\tutor}{Nicolas Lehmann}						% <-- Name eurer Tutoriun eintragen
\newcommand{\tutoriumNo}{10}				% <-- Nummer im KVV nachschauen
\newcommand{\projectNo}{5}									% <-- Nummer des Übungszettels
\newcommand{\veranstaltung}{Datenbanksysteme}	% <-- Name der Lehrveranstaltung eintragen
\newcommand{\semester}{SoSe 2017}						% <-- z.B. SoSe 17, WiSe 17/18
\newcommand{\studenten}{Boyan Hristov, Julian Habib}			% <-- Hier eure Namen eintragen
% /////////////////////// BEGIN DOKUMENT /////////////////////////


\begin{document}
% /////////////////////// BEGIN TITLEPAGE /////////////////////////
\begin{titlepage}
	\subject{\dozent}
	\title{\veranstaltung, \semester}
	\subtitle{\Large Übungsblatt \projectNo\\ \large\vspace{1ex} }
	\author{\studenten}
	\date{\normalsize \today}
\end{titlepage}

\maketitle								% Erstellt das Titelblatt
\vspace*{-9cm}							% rückt Logo an den oberen Seitenrand
\makebox[\dimexpr\textwidth+1cm][r]{	%rechtsbündig und geht rechts 1cm über Layout hinaus
	\includegraphics[width=0.4\textwidth]{src/fu_logo} % fügt FU-Logo ein
}
% /////////////////////// END TITLEPAGE /////////////////////////

\vspace{7cm}							% Abstand
\rule{\linewidth}{0.8pt}				% horizontale Linie										% erstellt die Titelseite


Link zum Git Repository: \url{https://github.com/BoyanH/Freie-Universitaet-Berlin/tree/master/Datenbanksysteme/Solutions/homework5}

% /////////////////////// Aufgabe 1 /////////////////////////
\section*{1. Aufgabe}

\begin{itemize}

\item[a)]

\begin{align*}
\text{Minimal F} = \{ & \\
& \{B\} \rightarrow \{CD\} \\
& \{C\} \rightarrow \{E\} \\
& \{D\} \rightarrow \{E\} \\
& \{AE\} \rightarrow \{D\} \\
\} & \\ \\
& \Rightarrow \\
F+ = \{ & \\
& \{B\} \rightarrow \{CD\} \\
& \{C\} \rightarrow \{E\} \\
& \{D\} \rightarrow \{E\} \\
& \{AE\} \rightarrow \{D\} \\
& \{B\} \rightarrow \{C\} \\
& \{B\} \rightarrow \{D\} \\
& \{BC\} \rightarrow \{CDE\} \\
& \{BD\} \rightarrow \{CDE\} \\
& \{CD\} \rightarrow \{E\} \\
& \{AB\} \rightarrow \{CDE\} \\
& ... \\
\} &
\end{align*}

Damit ist AB ein Key / Schlüssel.

\item[b)]
Pseudotransitivität wäre hier
\begin{align*}
& \{C\} \rightarrow \{E\} \land \{AE\} \rightarrow D \\
\Rightarrow & \{AC\} \rightarrow \{D\}
\end{align*}

\item[c)]
\begin{align*}
& \{BE\} \rightarrow \{B\} \text{(Reflexivity)} \\
\Rightarrow & \{A\} \rightarrow \{BE\} \land \{BE\} \rightarrow \{B\} \\
\Rightarrow & \{A\} \rightarrow \{B\} (Transitivity) \\
\Rightarrow & \{A\} \rightarrow \{B\} \land \{A\} \rightarrow \{A\} \text{(Reflexivity)} \\
\Rightarrow & \{A\} \rightarrow \{AB\} \text{(Union)} \land \{AB\} \rightarrow \{C\} \\
\Rightarrow & \{A\} \rightarrow \{C\} \text{(Transitivity)} \\
\Rightarrow & \{A\} \rightarrow \{C\} \land \{C\} \rightarrow \{D\} \\
\Rightarrow & \{A\} \rightarrow \{D\} \text{(Transitivity)}
\end{align*}

\end{itemize}

\section*{2. Aufgabe}

\begin{itemize}

\item[a)]



\end{itemize}



% /////////////////////// END DOKUMENT /////////////////////////
\end{document}
