\input{src/header}											% bindet Header ein (WICHTIG)
\usepackage{graphicx}
\usepackage{amsmath}
\usepackage{amssymb}
\usepackage{fancyvrb}

\newcommand{\dozent}{Prof. Dr. Margarita Esponda}					% <-- Names des Dozenten eintragen
\newcommand{\tutor}{Lilli Walter}						% <-- Name eurer Tutoriun eintragen
\newcommand{\tutoriumNo}{6}				% <-- Nummer im KVV nachschauen
\newcommand{\projectNo}{8}									% <-- Nummer des Übungszettels
\newcommand{\veranstaltung}{Nichtsequentielle Programmierung}	% <-- Name der Lehrveranstaltung eintragen
\newcommand{\semester}{SoeSe 2017}						% <-- z.B. SoSo 17, WiSe 17/18
\newcommand{\studenten}{Boyan Hristov, Sergelen Gongor}			% <-- Hier eure Namen eintragen
% /////////////////////// BEGIN DOKUMENT /////////////////////////


\begin{document}
% /////////////////////// BEGIN TITLEPAGE /////////////////////////
\begin{titlepage}
	\subject{\dozent}
	\title{\veranstaltung, \semester}
	\subtitle{\Large Übungsblatt \projectNo\\ \large\vspace{1ex} }
	\author{\studenten}
	\date{\normalsize \today}
\end{titlepage}

\maketitle								% Erstellt das Titelblatt
\vspace*{-9cm}							% rückt Logo an den oberen Seitenrand
\makebox[\dimexpr\textwidth+1cm][r]{	%rechtsbündig und geht rechts 1cm über Layout hinaus
	\includegraphics[width=0.4\textwidth]{src/fu_logo} % fügt FU-Logo ein
}
% /////////////////////// END TITLEPAGE /////////////////////////

\vspace{7cm}							% Abstand
\rule{\linewidth}{0.8pt}				% horizontale Linie										% erstellt die Titelseite


Link zum Git Repository: \url{https://github.com/BoyanH/FU-Berlin-ALP4/tree/master/Solutions/Homework\projectNo}

% /////////////////////// Aufgabe 1 /////////////////////////

\section*{1. Aufgabe}

\begin{enumerate}

\item[a)] TCP Echo Server

\begin{lstlisting}[style=java]
import java.net.*;
import java.io.*;

public class TCPEchoServer {

    public static void main(String[] args) {
    
        // not as it should be done, I hope enough for this exercise. Usually each error should be handled individually
        try {

            int port = 1337;
            String ipAddress = "localhost";
            SocketAddress address = new InetSocketAddress(ipAddress, port);
            ServerSocket socket = new ServerSocket(); // create unbound server socket
            socket.bind(address); // bind it to the given address (ip + port)

            System.out.println("TCP server running on localhost:1337");

            // keep accepting client connections
            while (true) {
                Socket connection = socket.accept(); // wait for client connections and accept them
                DataOutputStream out = new DataOutputStream(connection.getOutputStream());
                BufferedReader br = new BufferedReader(new InputStreamReader(connection.getInputStream()));
                String currentLine = br.readLine();

                System.out.print("Received message \"");
                // read each line from the client and write it back
                while (currentLine != null) {
                    System.out.print(currentLine);
                    out.writeBytes(currentLine + '\n');
                    currentLine = br.readLine();
                }
                System.out.println("\" and wrote it back to client!");

                // close all streams and connections when finished
                out.close();
                br.close();
                connection.close();
            }

        } catch (IOException e) {
            e.printStackTrace();
        }
    }
}

\end{lstlisting}

\item[b)] TCP Echo Client

\begin{lstlisting}[style=java]
import java.net.*;
import java.io.*;

public class TCPEchoClient {

    public static void main(String[] args) {
    
        try {

            int port = 1337;
            String ipAddress = "localhost";
            Socket socket = new Socket(ipAddress, port); // create a client socket and bind it to our ServerSocket

            // Define all required IO streams and variables required for reading/writing from/to server
            DataOutputStream out;
            BufferedReader stdin;
            BufferedReader serverIn;
            String currentLine;
            String serverResponseLine;

            System.out.println("TCP client connected to localhost:1337"); // after new Socket() call has executed, client should be connected

            // initialize IO buffers
            out = new DataOutputStream(socket.getOutputStream());
            stdin = new BufferedReader(new InputStreamReader(System.in));
            serverIn = new BufferedReader(new InputStreamReader(socket.getInputStream()));

            System.out.print("Message to send to server: ");
            currentLine = stdin.readLine(); // read from console (stdin)
            out.writeBytes(currentLine + '\n'); // write to server
            serverResponseLine = serverIn.readLine(); // read response from server
            System.out.printf("Server response: %s\n", serverResponseLine);

            // close all buffers and socket connection
            out.close();
            stdin.close();
            serverIn.close();
            socket.close();
            


        } catch (IOException e) {
            e.printStackTrace();
        }
    }
}

\end{lstlisting}

\item[c)] UDP Echo Server

\begin{lstlisting}[style=java]
import java.net.*;
import java.io.*;

public class UDPEchoServer {

    public static void main(String[] args) {
    
        // not as it should be done, I hope enough for this exercise. Usually each error should be handled individually
        try {

            int port = 1337;
            InetAddress laddr = InetAddress.getLocalHost();
            byte[] inDataBuffer = new byte[256];
            DatagramSocket udSocket = new DatagramSocket(1337, laddr);

            DatagramPacket udClientPacket;
            DatagramPacket udResponsePacket;
            int clientPort;
            InetAddress clientAddress;
            String clientMessage;

            System.out.println("Started UDP server on localhost:1337");

            // keep accepting client connections
            while (true) {
                udClientPacket = new DatagramPacket(inDataBuffer, inDataBuffer.length);
                udSocket.receive(udClientPacket); // wait to receive client user-datagramm-packet
                clientMessage = new String(udClientPacket.getData()); // get String from send byte[]
                clientAddress = udClientPacket.getAddress();
                clientPort = udClientPacket.getPort();

                udResponsePacket = new DatagramPacket(inDataBuffer, inDataBuffer.length, clientAddress, clientPort);
                udSocket.send(udResponsePacket);

                System.out.printf("Received message \"%s\" and wrote it back to client.", clientMessage);

                // nothing to close with UDP (except for the server socket, in our case NEVER! ^^)
            }

        } catch (IOException e) {
            e.printStackTrace();
        }
    }
}

\end{lstlisting}

\item[d)] UDP Echo Client

\begin{lstlisting}[style=java]
import java.net.*;
import java.io.*;

public class UDPEchoClient {

    public static void main(String[] args) {
    
        // not as it should be done, I hope enough for this exercise. Usually each error should be handled individually
        try {

            int port = 1337;
            InetAddress laddr = InetAddress.getLocalHost();
            byte[] responseDataBuffer = new byte[256];
            byte[] sendBuffer;
            DatagramSocket udSocket = new DatagramSocket(); // create new unbound datagramm socket

            DatagramPacket udClientPacket;
            DatagramPacket udResponsePacket = new DatagramPacket(responseDataBuffer, responseDataBuffer.length);
            BufferedReader stdin = new BufferedReader(new InputStreamReader(System.in));

            // read from stdin
            System.out.print("Enter message to send to server: ");
            sendBuffer = stdin.readLine().getBytes();

            // create a DatagramPacket from the read message and send to server
            udClientPacket = new DatagramPacket(sendBuffer, sendBuffer.length, laddr, port);
            udSocket.send(udClientPacket);

            // read server response
            udSocket.receive(udResponsePacket);
            System.out.printf("Server response: %s\n", new String(udResponsePacket.getData()));

            udSocket.close();
        } catch (IOException e) {
            e.printStackTrace();
        }
    }
}

\end{lstlisting}

\end{enumerate}

\section*{Aufgabe 2}
Wir haben alle Aufgaben für das TCP Chat bearbeitet, alle Features sollen funktionieren.

\begin{enumerate}

\item Client \\

\begin{lstlisting}[style=java]
package network;

import fx.ClientGUI;
import javafx.scene.paint.Color;

import java.io.IOException;
import java.io.PrintWriter;
import java.net.Socket;

public class ClientTCP extends AbstractChatClient {

    private MessageListenerTCP messageListenerTCP;
    private boolean clientConnected;
    private Socket serverSocket;
    private PrintWriter socketWriter;

    public ClientTCP(ClientGUI gui) {
        super(gui);
    }

    @Override
    public void sendChatMessage(String msg) {
        if (this.socketWriter != null) {
            this.socketWriter.println(msg);
        }
    }

    @Override
    public void connect(String address, String port) {
        if (this.clientConnected) {
            return;
        }

        boolean connected = true;
        int portNumber;

        try {

            portNumber = Integer.parseInt(port);
            this.serverSocket = new Socket(address, portNumber);
            this.socketWriter = new PrintWriter(this.serverSocket.getOutputStream(), true); // auto-flush output stream
            this.connectToChat();

            this.messageListenerTCP = new MessageListenerTCP(this.serverSocket, this);
            this.messageListenerTCP.start();


        } catch (NumberFormatException e) {
            this.gui.pushChatMessage("System: Port must be a valid integer!");
            connected = false;
        } catch (IOException e) {
            this.terminate();
        }

        this.clientConnected = connected;
    }

    @Override
    public void disconnect() {
        this.terminate();
    }

    @Override
    public void terminate() {
        this.setConnected(false);

        if (this.messageListenerTCP != null) {
            this.messageListenerTCP.terminate();
        }
        try {
            if (this.serverSocket != null) {
                this.serverSocket.close();
            }
            if (this.socketWriter != null) {
                this.socketWriter.close();
            }
        } catch (IOException e) {
            e.printStackTrace();
        }
    }

    @Override
    public void setUName(String name) {
        super.setUName(name);

        // notify server
        this.socketWriter.printf("/n %s\n", this.getUName());
    }

    public void setConnected(boolean connected) {
        this.clientConnected = connected;
        this.gui.setSymbolColor(this.clientConnected ? Color.GREEN : Color.RED);
    }

    public void onServerMessage(String message) {
        if (message.charAt(0) == '/') {
            this.handleServerCommand(message);
        } else {
            this.gui.pushChatMessage(message);
        }
    }

    public void connectToChat() {
        if (this.socketWriter != null) {
            this.socketWriter.printf("/n %s\n", this.getUName());
            this.socketWriter.flush();
            this.setConnected(true);
        }
    }

    private void handleServerCommand(String command) {
        switch(command.substring(0,2)) {
            case "/r":
                int firstSpaceIdx = command.indexOf(' ');
                int secondSpaceIdx = command.indexOf(' ', firstSpaceIdx + 1);
                this.gui.pushChatMessage(
                        String.format("%s renamed to %s",
                                command.substring(firstSpaceIdx + 1, secondSpaceIdx),
                                command.substring(secondSpaceIdx + 1)
                        )
                );
                break;
            default:
                System.out.println("Unrecognized server command!");
        }
    }
}

\end{lstlisting}

\begin{lstlisting}[style=java]
package network;

import fx.ClientGUI;
import javafx.scene.paint.Color;

import java.io.BufferedReader;
import java.io.IOException;
import java.io.InputStreamReader;
import java.net.*;

public class MessageListenerTCP extends Thread {

    private final int MAX_CHAT_REFRESH_LENGTH = 4096;
    private String serverAddress;
    private int serverPort;
    private Socket serverSocket;
    private boolean running = true;
    private ClientTCP client;
    private BufferedReader socketBuffer;


    public MessageListenerTCP(Socket serverSocket, ClientTCP client) throws NumberFormatException {
        this.serverSocket = serverSocket;
        this.client = client;

        try {
            this.socketBuffer = new BufferedReader(new InputStreamReader(this.serverSocket.getInputStream()));
            this.socketBuffer.mark(MAX_CHAT_REFRESH_LENGTH);
        } catch (IOException e) {
            this.terminate();
            return;
        }
    }

    public BufferedReader getReader() {
        return this.socketBuffer;
    }

    public void terminate() {
        this.running = false;
        this.client.setConnected(false);
    }

    public Socket getServerSocket() {
        return this.serverSocket;
    }

    @Override
    public void run() {
        this.waitForMessages();
    }

    /**
     * Waits for server to send a message. If the new message is null, then the server has close the
     * connection and we need to notify the client. If a new message is received, call ClientTCP::onServerMessage()
     */
    public void waitForMessages() {
        String inputLine;

        try {
            this.serverSocket.setSoTimeout(1000);
        } catch (SocketException e) {
            e.printStackTrace();
        }

        while (this.running) {
            try {
                inputLine = socketBuffer.readLine();
                if (inputLine == null) {
                    // server not available
                    this.terminate();
                } else {
                    this.client.onServerMessage(inputLine);
                }
            } catch (SocketTimeoutException e) {
                // timed out, server is there, has nothing to say
            } catch (IOException e) {
                this.terminate();
            }
        }
    }
}

\end{lstlisting}

\item Server \\

\begin{lstlisting}[style=java]
package network;

import fx.ServerGUI;
import javafx.scene.paint.Color;

import java.io.BufferedReader;
import java.io.IOException;
import java.net.*;
import java.util.LinkedList;
import java.util.List;

public class ServerTCP extends AbstractChatServer {

    private boolean running = false;
    private ServerSocket serverSocket;
    private ConnectionAccepterTCP connectionAccepter;
    private List<ClientCommunicationThread> clientMessagers;

    public ServerTCP(ServerGUI gui) {
        super(gui);
        this.clientMessagers = new LinkedList<ClientCommunicationThread>();
    }

    @Override
    public void receiveConsoleCommand(String command, String msg) {
        switch(command) {
            case "/exclude":
                this.removeClientByName(msg);
                break;
            case "/mute":
                String[] words = msg.split(" ");
                if (words.length < 2) {
                    this.gui.pushConsoleMessage("Invalid number of arguments for command /mute");
                    return;
                }
                this.muteClientByNameForSeconds(words[0], words[1]);
                break;
            default:
                this.gui.pushConsoleMessage(String.format("Unrecognized console command %s", command));
        }

    }

    @Override
    public void start(String port) {
        SocketAddress address;
        int portNumber;

        if (this.running) {
            return;
        }

        try {
            portNumber = Integer.parseInt(port);
        } catch (NumberFormatException e) {
            System.out.println("Port must be an integer!");
            return;
        }
        address = new InetSocketAddress("localhost", portNumber);
        try {
            this.serverSocket = new ServerSocket(); // create unbound server socket
            this.serverSocket.bind(address);
            this.running = true;
            this.gui.setSymbolColor(Color.GREEN);
            this.connectionAccepter = new ConnectionAccepterTCP(this.serverSocket, this);
            this.connectionAccepter.start();

        } catch (IOException e) {
            e.printStackTrace();
            this.terminate();
            return;
        }

    }

    @Override
    public void stop() {
        this.running = false;
        this.gui.setSymbolColor(Color.RED);

        if (this.connectionAccepter != null) {
            this.connectionAccepter.terminate();
        }

        try {
            this.serverSocket.close();
        } catch (IOException e) {
            e.printStackTrace();
        }

        this.removeAllClients();

    }

    @Override
    public void terminate() {
        this.stop();

    }

    public void addNewClient(Socket connection) {
        ClientCommunicationThread newClientThread = new ClientCommunicationThread(connection, this);
        newClientThread.start();
        this.clientMessagers.add(newClientThread);
    }

    public void removeClient(ClientCommunicationThread client) {
        this.clientMessagers.remove(client);
        this.gui.removeClient(client.getClientId());
        client.terminate();
    }

    public void removeAllClients() {
        while(!this.clientMessagers.isEmpty()) {
            ClientCommunicationThread crnt = this.clientMessagers.remove(0);
            this.gui.removeClient(crnt.getClientId());
            crnt.terminate();
        }
    }

    public void onNewMessage(ClientCommunicationThread clientMessager, String message) {
        if (clientMessager.mutedBefore > System.currentTimeMillis()) {
            clientMessager.getWriter().println("You are muted!");
            clientMessager.getWriter().flush();
            return; // muted
        }

        for (ClientCommunicationThread client : this.clientMessagers) {
            client.getWriter().printf("%s: %s\n", clientMessager.getUName(), message);
            client.getWriter().flush();
        }
    }

    public void addClientToGUI(int id, String uname) {
        this.gui.addClient(id, uname);
    }

    public void clientRenamed(int id, String oldName, String newName) {
        this.gui.removeClient(id);
        this.gui.addClient(id, newName);

        // better send not as direct message but as command, wanted to
        // refresh all mesages, but the gui windows is not flushable ;/
        for (ClientCommunicationThread client : this.clientMessagers) {
            client.getWriter().printf("/r %s %s\n", oldName, newName);
            client.getWriter().flush();
        }
    }

    public void sendPrivateMessage(ClientCommunicationThread fromClient, String toClientName, String message) {
        fromClient.getWriter().printf("*private* %s: %s\n", fromClient.getUName(), message);
        fromClient.getWriter().flush();
        this.getClientByName(toClientName).getWriter().printf("*private* %s: %s\n", fromClient.getUName(), message);
        this.getClientByName(toClientName).getWriter().flush();
    }

    private ClientCommunicationThread getClientByName(String name) {
        for (ClientCommunicationThread client : this.clientMessagers) {
            if (client.getUName().equals(name)) {
                return client;
            }
        }

        return  null;
    }

    private void removeClientByName(String name) {
        this.removeClient(this.getClientByName(name));
    }

    private void muteClientByNameForSeconds(String name, String seconds) {
        int milliseconds;

        try {
            milliseconds = Integer.parseInt(seconds) * 1000;
            this.getClientByName(name).mutedBefore = System.currentTimeMillis() + milliseconds;
        } catch (NumberFormatException e) {
            this.gui.pushConsoleMessage("Invalid argument for seconds to mute parsed!");
        }
    }

}

\end{lstlisting}

\begin{lstlisting}[style=java]
package network;

import java.io.BufferedReader;
import java.io.IOException;
import java.io.InputStreamReader;
import java.net.ServerSocket;
import java.net.Socket;
import java.net.SocketException;
import java.net.SocketTimeoutException;
import java.util.LinkedList;

public class ConnectionAccepterTCP extends Thread {

    private ServerSocket serverSocket;
    private ServerTCP server;
    private boolean running = true;
    private LinkedList<Socket> connections;

    public ConnectionAccepterTCP(ServerSocket socket, ServerTCP server) {
        this.serverSocket = socket;
        this.server = server;
        this.connections = new LinkedList<>();
    }

    @Override
    public void run() {
        this.waitForConections();
    }

    public void waitForConections() {
        String inputLine;
        BufferedReader socketBuffer = null;

        while (this.running) {
            try {
                Socket connection = this.serverSocket.accept(); // wait for client connections and accept them

                if (!this.connections.contains(connection)) {
                    this.server.addNewClient(connection);
                    this.connections.add(connection);
                }

            } catch (IOException e) {
                // socket setTimeout exception, whatever
            }
        }
    }

    public void terminate() {
        this.running = false;
    }
}

\end{lstlisting}

\begin{lstlisting}[style=java]
package network;

import java.io.BufferedReader;
import java.io.IOException;
import java.io.InputStreamReader;
import java.io.PrintWriter;
import java.net.Socket;
import java.net.SocketException;
import java.net.SocketTimeoutException;

public class ClientCommunicationThread extends Thread {

    static int clientsCount = 0;

    private int id;
    private Socket clientConnection;
    private String clientName;
    private boolean running = true;
    private ServerTCP server;
    private PrintWriter socketWriter;
    public long mutedBefore = System.currentTimeMillis();

    public ClientCommunicationThread(Socket connection, ServerTCP server) {
        this.id = ++clientsCount;
        this.clientConnection = connection;
        this.server = server;
        try {
            this.socketWriter = new PrintWriter(this.clientConnection.getOutputStream(), true); // auto-flush output stream
        } catch (IOException e) {
            e.printStackTrace();
            this.terminate();
        }
    }

    public String getUName() {
        return this.clientName;
    }

    public void setUName(String name) {
        if (this.clientName == null) {
            this.server.addClientToGUI(this.id, name);
        } else {
            this.server.clientRenamed(this.getClientId(), this.clientName, name);
        }
        this.clientName = name;
    }

    public PrintWriter getWriter() {
        return this.socketWriter;
    }

    public int getClientId() {
        return this.id;
    }

    @Override
    public void run() {
        this.waitForMessages();
    }

    public void terminate() {
        this.running = false;
        try {
            this.clientConnection.close();
            this.socketWriter.close();
        } catch (IOException e) {
            e.printStackTrace();
        }

    }

    public void waitForMessages() {
        String inputLine;
        BufferedReader socketBuffer;

        try {
            socketBuffer = new BufferedReader(new InputStreamReader(this.clientConnection.getInputStream()));
            socketBuffer.mark(4096);
        } catch (IOException e) {
            e.printStackTrace();
            this.terminate();
            return;
        }

        try {
            this.clientConnection.setSoTimeout(1000);
        } catch (SocketException e) {
            e.printStackTrace();
        }

        while (this.running) {
            try {
                inputLine = socketBuffer.readLine();
                if (inputLine == null) {
                    // client disconnected
                    this.server.removeClient(this);
                } else if (inputLine.charAt(0) == '/') {
                    this.parseClientCommand(inputLine);
                } else {
                    this.server.onNewMessage(this, inputLine);
                }
            } catch (SocketTimeoutException e) {
                // timed out, client is there, has nothing to say
            } catch (IOException e) {
                this.server.removeClient(this);
            }
        }

        try {
            socketBuffer.close();
        } catch (IOException e) {
            e.printStackTrace();
        }
    }

    public void parseClientCommand(String clientMessage) {;
        int firstSpaceIndex = clientMessage.indexOf(' ');
        String command = clientMessage.substring(0, firstSpaceIndex);
        String rest = clientMessage.substring(firstSpaceIndex+1);

        switch(command) {
            case "/n":
                this.setUName(rest);
                break;
            case "/private":
                this.handlePrivateMessage(rest);
            default:
                // unrecognized command
                break;
        }
    }

    private void handlePrivateMessage(String message) {
        int firstSpaceIdx = message.indexOf(' ');
        String uName = message.substring(0, firstSpaceIdx);
        String privateMessage = message.substring(firstSpaceIdx+1);
        this.server.sendPrivateMessage(this, uName, privateMessage);
    }
}

\end{lstlisting}

\end{enumerate}


% /////////////////////// END DOKUMENT /////////////////////////
\end{document}