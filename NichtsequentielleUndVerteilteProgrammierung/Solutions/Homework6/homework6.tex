\input{src/header}											% bindet Header ein (WICHTIG)
\usepackage{graphicx}
\usepackage{amsmath}
\usepackage{amssymb}
\usepackage{fancyvrb}

\newcommand{\dozent}{Prof. Dr. Margarita Esponda}					% <-- Names des Dozenten eintragen
\newcommand{\tutor}{Lilli Walter}						% <-- Name eurer Tutoriun eintragen
\newcommand{\tutoriumNo}{6}				% <-- Nummer im KVV nachschauen
\newcommand{\projectNo}{5}									% <-- Nummer des Übungszettels
\newcommand{\veranstaltung}{Nichtsequentielle Programmierung}	% <-- Name der Lehrveranstaltung eintragen
\newcommand{\semester}{SoeSe 2017}						% <-- z.B. SoSo 17, WiSe 17/18
\newcommand{\studenten}{Boyan Hristov, Sergelen Gongor}			% <-- Hier eure Namen eintragen
% /////////////////////// BEGIN DOKUMENT /////////////////////////


\begin{document}
% /////////////////////// BEGIN TITLEPAGE /////////////////////////
\begin{titlepage}
	\subject{\dozent}
	\title{\veranstaltung, \semester}
	\subtitle{\Large Übungsblatt \projectNo\\ \large\vspace{1ex} }
	\author{\studenten}
	\date{\normalsize \today}
\end{titlepage}

\maketitle								% Erstellt das Titelblatt
\vspace*{-9cm}							% rückt Logo an den oberen Seitenrand
\makebox[\dimexpr\textwidth+1cm][r]{	%rechtsbündig und geht rechts 1cm über Layout hinaus
	\includegraphics[width=0.4\textwidth]{src/fu_logo} % fügt FU-Logo ein
}
% /////////////////////// END TITLEPAGE /////////////////////////

\vspace{7cm}							% Abstand
\rule{\linewidth}{0.8pt}				% horizontale Linie										% erstellt die Titelseite


Link zum Git Repository: \url{https://github.com/BoyanH/FU-Berlin-ALP4/tree/master/Solutions/Homework5}

% /////////////////////// Aufgabe 1 /////////////////////////

\section*{Aufgabe 1}

Zu zeigen: Await-Bedingung für $P_i \equiv B \equiv inD \leq afterF$
\begin{align*}
    wp(inD \leftarrow inD + 1, PCI) = & 
    wp(inD \leftarrow inD + 1, (inD \leq afterF + 1) \land (inF \leq afterD)) = \\
    = & (inD+1 \leq afterF + 1) \land (inF \leq afterD) \tag{Zuweisungsregel}\\
    = & (inD \leq afterF) \land (inF \leq afterD) \\ \\
    %
    \text{Da es } P \land INV \land B & \Rightarrow wp(S, Q \land INV) \text{ gelten muss ist hier } B = inD \leq afterF \\
    & \tag*{$\Box$}
\end{align*}

Zu zeigen: Await-Bedingung für $C_i \equiv B \equiv inF < afterD$
\begin{align*}
    wp(inF \leftarrow inF + 1, PCI) = & 
    wp(inF \leftarrow inF + 1, (inD \leq afterF + 1) \land (inF \leq afterD)) = \\
    = & (inD \leq afterF + 1) \land (inF + 1 \leq afterD) = \tag{Zuweisungsregel} \\
    = & (inD \leq afterF + 1) \land (inF < afterD) = \\ \\
    %
    \text{Da es } P \land INV \land B & \Rightarrow wp(S, Q \land INV) \text{ gelten muss ist hier } B = inF < afterD \\
    & \tag*{$\Box$}
\end{align*}

\section*{Aufgabe 2}

Leider war das Pseudo-Code, das in der Vorlesungsfolien stand, nicht richtig. Wir haben es repariert und danach die Bedingung für die Lese verändert -> diese können frei reinkommen ohne warten nur wenn es keine wartende Schreiber gibt und auch nur dann können sie weitere Leser reinlassen.

\begin{lstlisting}[style=java]
package fu.alp4;

public class Writer extends IDataUser {

    public void run() {
        while (true) {
            try {
                // take a random rest to simulate different scenarios
                randomNap(5000, 10000);

                E.acquire();

                /**
                 * If there are currently some other writers or readers, wait
                 * for an available writer spot to be freed from a reader
                 * In te given time, release E, but remember to acquire it before
                 * incrementing nw++ for synchronization
                 */
                if (nw > 0 || nr > 0) {
                    dw++;
                    System.out.println("Waits to start writing! Stop letting further readers!");
                    E.release();
                    W.acquire();
                    E.acquire();
                }

                nw++;
                System.out.printf("Started writing; nr: %s; nw: %s; dr: %s; dw: %s\n", nr, nw, dr, dw);
                E.release();

                randomNap(2000, 4000);

                E.acquire();
                nw--;

                if (dr > 0 && dw == 0) {
                    dr--;
                    R.release();
                } else if (dw > 0) {
                    /**
                     * Deferred writers have higher priority, because we thought it's important to
                     * let writers as soon as possible so readers get the latest and greatest ^^
                     *
                     * E.g if both writers and readers are waiting, let the writer in
                     */

                    dw--;
                    W.release();
                }
                System.out.printf("Finished writing; nr: %s; nw: %s; dr: %s; dw: %s\n", nr, nw, dr, dw);
                E.release();


            } catch (InterruptedException e) {
                e.printStackTrace();
            }
        }
    }
}

\end{lstlisting}

\begin{lstlisting}[style=java]
package fu.alp4;

public class Reader extends IDataUser {

    public void run() {
        while (true) {
            // take a random rest to simulate different scenarios
            randomNap(500, 2000);
            try {
                E.acquire();
                // skip waiting only if there are no deferred or non-deferred writers!
                // if a writer is waiting, he has the priority
                if (nw > 0 || dw > 0) {
                    dr++;
                    E.release();
                    R.acquire();
                    E.acquire();
                }

                nr++;
                // again, waiting writers have priority, don't let further readers in this case
                if (dr > 0 && dw == 0) {
                    dr--;
                    R.release();
                }

                System.out.printf("Started reading; nr: %s; nw: %s; dr: %s; dw: %s\n", nr, nw, dr, dw);
                E.release();

                // read
                randomNap(500, 2000);


                E.acquire();
                nr--;
                if (nr == 0 && dw > 0) {
                    dw--;
                    W.release();
                }

                System.out.printf("Finished reading; nr: %s; nw: %s; dr: %s; dw: %s\n", nr, nw, dr, dw);
                E.release();

            } catch (InterruptedException e) {
                e.printStackTrace();
            }
        }
    }
}
\end{lstlisting}


\begin{lstlisting}[style=java]
package fu.alp4;

public class Main {

    public static void main(String[] args) {

        for (int i = 0; i < 5; i++) {
            if (i < 4) {
                new Reader().start();
            } else {
                new Writer().start();
            }
        }
    }
}

\end{lstlisting}

\section*{Aufgabe 3}

\begin{enumerate}

\item[a)]

$ \sum_{n=1}^{n} C_{ij} + A_j = E_j$ wobei C ist die Belegungsmatrix, A der Ressourcenrestvektor, E der Ressourcenvektor (Aus Vorlesungsfolien). Also an dem Beispiel $ \sum_{n=1}^{n} C_{ij} + R_j = E_j$ wobei R der Ressourcenrestvektor ist.
    
\begin{align*}
    %
    \Rightarrow R_j & = E_j - \sum_{n=1}^{n} C_{ij} \\
    \Rightarrow R & = [(3 - (1+1)), (15 - (1+3+6)), (12 - (1+5+3+1)), (11 - (1+4+2+4))] = \\
    & = [1, 5, 2, 0]
    %
\end{align*}

Der Ressourcenrestvektor bzw. die noch vorhande Ressourcen R = [1, 5, 2, 0]

\item[b)]
Das System befindet sich in einem sicheren Zustand, weil es eine Scheduling-Reihenfolge gibt, die nicht zu Deadlock führt. Solche Reihenfolge ist z.B:

\begin{align*}
%
    & T_4 \Rightarrow R = [1,11,5,2] \\
    \rightarrow & T_5 \Rightarrow R = [1,11,6,6] \\
    \rightarrow & T_1 \Rightarrow R = [1,11,7,7] \\
    \rightarrow & T_2 \Rightarrow R = [2,12,6,6] \\
    \rightarrow & T_3 \Rightarrow R = [3,15,12,11] \\ \\
    & R = E \Rightarrow \text{ sicheren Zustand}
%
\end{align*}

\item[c)]

Da R = [1,5,2,0] wird nach dem Teilanforderung von Thread $T_2$ R = [1,2,0,0]. Das ist kein sicheren Zustand, da alle Threads danach (inklusive $T_2$ mit seinem neuen Restanforderung von [0,3,3,0]) mindestens eins von den letzten zwei Ressourcen brauchen, es gibt aber keine mehr vorhanden. Deswegen gibt es auch keine Scheduling-Reihenfolge, die nicht zu einem Deadlock führt.

$\Rightarrow$ soll nicht bedient werden.

\end{enumerate}

% /////////////////////// END DOKUMENT /////////////////////////
\end{document}