\input{src/header}											% bindet Header ein (WICHTIG)
\usepackage{graphicx}
\usepackage{amsmath}
\usepackage{amssymb}
\usepackage{fancyvrb}

\newcommand{\dozent}{Prof. Dr. Margarita Esponda}					% <-- Names des Dozenten eintragen
\newcommand{\tutor}{Lilli Walter}						% <-- Name eurer Tutoriun eintragen
\newcommand{\tutoriumNo}{6}				% <-- Nummer im KVV nachschauen
\newcommand{\projectNo}{5}									% <-- Nummer des Übungszettels
\newcommand{\veranstaltung}{Nichtsequentielle Programmierung}	% <-- Name der Lehrveranstaltung eintragen
\newcommand{\semester}{SoeSe 2017}						% <-- z.B. SoSo 17, WiSe 17/18
\newcommand{\studenten}{Boyan Hristov, Sergelen Gongor}			% <-- Hier eure Namen eintragen
% /////////////////////// BEGIN DOKUMENT /////////////////////////


\begin{document}
% /////////////////////// BEGIN TITLEPAGE /////////////////////////
\begin{titlepage}
	\subject{\dozent}
	\title{\veranstaltung, \semester}
	\subtitle{\Large Übungsblatt \projectNo\\ \large\vspace{1ex} }
	\author{\studenten}
	\date{\normalsize \today}
\end{titlepage}

\maketitle								% Erstellt das Titelblatt
\vspace*{-9cm}							% rückt Logo an den oberen Seitenrand
\makebox[\dimexpr\textwidth+1cm][r]{	%rechtsbündig und geht rechts 1cm über Layout hinaus
	\includegraphics[width=0.4\textwidth]{src/fu_logo} % fügt FU-Logo ein
}
% /////////////////////// END TITLEPAGE /////////////////////////

\vspace{7cm}							% Abstand
\rule{\linewidth}{0.8pt}				% horizontale Linie										% erstellt die Titelseite


Link zum Git Repository: \url{https://github.com/BoyanH/FU-Berlin-ALP4/tree/master/Solutions/Homework5}

% /////////////////////// Aufgabe 1 /////////////////////////

\section*{Aufgabe 1}

Zu zeigen: Await-Bedingung für $P_i \equiv B \equiv inD \leq afterF$
\begin{align*}
    wp(inD \leftarrow inD + 1, PCI) = & 
    wp(inD \leftarrow inD + 1, (inD \leq afterF + 1) \land (inF \leq afterD)) = \\
    = & (inD+1 \leq afterF + 1) \land (inF \leq afterD) \tag{Zuweisungsregel}\\
    = & (inD \leq afterF) \land (inF \leq afterD) \\ \\
    %
    \text{Da es } P \land INV \land B & \Rightarrow wp(S, Q \land INV) \text{ gelten muss ist hier } B = inD \leq afterF \\
    & \tag*{$\Box$}
\end{align*}

Zu zeigen: Await-Bedingung für $C_i \equiv B \equiv inF < afterD$
\begin{align*}
    wp(inF \leftarrow inF + 1, PCI) = & 
    wp(inF \leftarrow inF + 1, (inD \leq afterF + 1) \land (inF \leq afterD)) = \\
    = & (inD \leq afterF + 1) \land (inF + 1 \leq afterD) = \tag{Zuweisungsregel} \\
    = & (inD \leq afterF + 1) \land (inF < afterD) = \\ \\
    %
    \text{Da es } P \land INV \land B & \Rightarrow wp(S, Q \land INV) \text{ gelten muss ist hier } B = inF < afterD \\
    & \tag*{$\Box$}
\end{align*}

\section*{Aufgabe 2}

Leider war das Pseudo-Code, das in der Vorlesungsfolien stand, nicht richtig. Wir haben es repariert und danach die Bedingung für die Lese verändert -> diese können frei reinkommen ohne warten nur wenn es keine wartende Schreiber gibt und auch nur dann können sie weitere Leser reinlassen.

\begin{lstlisting}[style=java]
package fu.alp4;

public class Writer extends IDataUser {

    public void run() {
        while (true) {
            try {
                // take a random rest to simulate different scenarios
                randomNap(5000, 10000);

                E.acquire();

                /**
                 * If there are currently some other writers or readers, wait
                 * for an available writer spot to be freed from a reader
                 * In te given time, release E, but remember to acquire it before
                 * incrementing nw++ for synchronization
                 */
                if (nw > 0 || nr > 0) {
                    dw++;
                    System.out.println("Waits to start writing! Stop letting further readers!");
                    E.release();
                    W.acquire();
                    E.acquire();
                }

                nw++;
                System.out.printf("Started writing; nr: %s; nw: %s; dr: %s; dw: %s\n", nr, nw, dr, dw);
                E.release();

                randomNap(2000, 4000);

                E.acquire();
                nw--;

                if (dr > 0 && dw == 0) {
                    dr--;
                    R.release();
                } else if (dw > 0) {
                    /**
                     * Deferred writers have higher priority, because we thought it's important to
                     * let writers as soon as possible so readers get the latest and greatest ^^
                     *
                     * E.g if both writers and readers are waiting, let the writer in
                     */

                    dw--;
                    W.release();
                }
                System.out.printf("Finished writing; nr: %s; nw: %s; dr: %s; dw: %s\n", nr, nw, dr, dw);
                E.release();


            } catch (InterruptedException e) {
                e.printStackTrace();
            }
        }
    }
}

\end{lstlisting}

\begin{lstlisting}[style=java]
package fu.alp4;

public class Reader extends IDataUser {

    public void run() {
        while (true) {
            // take a random rest to simulate different scenarios
            randomNap(500, 2000);
            try {
                E.acquire();
                // skip waiting only if there are no deferred or non-deferred writers!
                // if a writer is waiting, he has the priority
                if (nw > 0 || dw > 0) {
                    dr++;
                    E.release();
                    R.acquire();
                    E.acquire();
                }

                nr++;
                // again, waiting writers have priority, don't let further readers in this case
                if (dr > 0 && dw == 0) {
                    dr--;
                    R.release();
                }

                System.out.printf("Started reading; nr: %s; nw: %s; dr: %s; dw: %s\n", nr, nw, dr, dw);
                E.release();

                // read
                randomNap(500, 2000);


                E.acquire();
                nr--;
                if (nr == 0 && dw > 0) {
                    dw--;
                    W.release();
                }

                System.out.printf("Finished reading; nr: %s; nw: %s; dr: %s; dw: %s\n", nr, nw, dr, dw);
                E.release();

            } catch (InterruptedException e) {
                e.printStackTrace();
            }
        }
    }
}
\end{lstlisting}


\begin{lstlisting}[style=java]
package fu.alp4;

public class Main {

    public static void main(String[] args) {

        for (int i = 0; i < 5; i++) {
            if (i < 4) {
                new Reader().start();
            } else {
                new Writer().start();
            }
        }
    }
}

\end{lstlisting}

\section*{Aufgabe 3}

\begin{enumerate}

\item[a)]

$ \sum_{n=1}^{n} C_{ij} + A_j = E_j$ wobei C ist die Belegungsmatrix, A der Ressourcenrestvektor, E der Ressourcenvektor (Aus Vorlesungsfolien). Also an dem Beispiel $ \sum_{n=1}^{n} C_{ij} + R_j = E_j$ wobei R der Ressourcenrestvektor ist.
    
\begin{align*}
    %
    \Rightarrow R_j & = E_j - \sum_{n=1}^{n} C_{ij} \\
    \Rightarrow R & = [(3 - (1+1)), (15 - (1+3+6)), (12 - (1+5+3+1)), (11 - (1+4+2+4))] = \\
    & = [1, 5, 2, 0]
    %
\end{align*}

Der Ressourcenrestvektor bzw. die noch vorhande Ressourcen R = [1, 5, 2, 0]

\item[b)]
Das System befindet sich in einem sicheren Zustand, weil es eine Scheduling-Reihenfolge gibt, die nicht zu Deadlock führt. Solche Reihenfolge ist z.B:

\begin{align*}
%
    & T_4 \Rightarrow R = [1,11,5,2] \\
    \rightarrow & T_5 \Rightarrow R = [1,11,6,6] \\
    \rightarrow & T_1 \Rightarrow R = [1,11,7,7] \\
    \rightarrow & T_2 \Rightarrow R = [2,12,6,6] \\
    \rightarrow & T_3 \Rightarrow R = [3,15,12,11] \\ \\
    & R = E \Rightarrow \text{ sicheren Zustand}
%
\end{align*}

\item[c)]

Da R = [1,5,2,0] wird nach dem Teilanforderung von Thread $T_2$ R = [1,2,0,0]. Das ist kein sicheren Zustand, da alle Threads danach (inklusive $T_2$ mit seinem neuen Restanforderung von [0,3,3,0]) mindestens eins von den letzten zwei Ressourcen brauchen, es gibt aber keine mehr vorhanden. Deswegen gibt es auch keine Scheduling-Reihenfolge, die nicht zu einem Deadlock führt.

$\Rightarrow$ soll nicht bedient werden.

\end{enumerate}

\section{Aufgabe 4}

\begin{enumerate}

\item[a)] Done

\item[b)] 
Der Algorithmus bearbeitet alle Requests nach einander, wobei er immer Ressourcen gibt, wenn solche vorhanden sind. Dafür speichert diese Lösung initial die freie Ressourcen, die von der Methode getFreeRes() in der Klasse Resources zurückgegeben werden und verändert immer die freie Ressources entsprechend. Da es ein Array ist werden die entsprechend auch in dem Objekt resources gesetzt.

Der Algorithmus funktioniert nur unter der Annahme, dass die gegebene Reihenfolge von Requests von links nach rechts abgearbeitet werden kann, ohne dass das System in einem Deadlock geht. Es ist praktisch kein Algorithmus, egal was die Threads wollen, wird es zu denen gleich gegeben, falls es genug Resourcen gibt, sonst nicht.

\item[c)]
\begin{enumerate}

\item[1.]

ExampleAlgo2 benutzt eine Warteschlange, wo neue Prozesse immer eingefügt werden, als diese kommen, und dann wieder entfernt werden, als diese fertig sind. So gibt der Algorithmus Ressources nur zu den Prozessen, die auf den ersten zwei Positionen in der Warteschlange sind und zwar nur wenn es genug freie Ressource gibt. Weiter wenn der 1 Prozess in der Warteschlange nie fertig wird, wird er auch nicht entfernt.

\item[2.]

Der Algorithmus funktioniert nur unter der Annahme richtig, dass zwei (oder bzw. auch 1) Prozess nicht so Ressourcen erfordern kann, dass es zu einem Deadlock kommt. Falls aber z.B es nicht genug Ressourcen für die erste zwei Prozessen in der Warteschlange gibt, dann kriegt man ein schönes Deadlock.

\item[3.]
Man kann immer schauen, ob die maximal erforderten Ressourcen von dem konkreten Prozess und dem letzten in der Warteschlange weniger als den freien Ressourcen sind. Da immer nur der erste Prozess in der Warteschlange entfernt wird ist man sicher, dass Prozesse, die nicht nebeneinander in der Warteschlange stehen, zusammen die Ressourcen sich teilen werden.

Eine konkrete Implementierung die in allen Fällen richtig funktioniert (nur die IF-Anweisung):

\begin{lstlisting}[style=java]
if(!q.contains(id) &&
                    (
                            (q.isEmpty() && CUtil.greaterEqual(free, req.getProcess().getPendingRes())) ||
                            (q.size() >= 1 && CUtil.greaterEqual(free, CUtil.add(processes[q.get(q.size() - 1)].getPendingRes(), req.getProcess().getPendingRes())))
                    )
              ) {

                q.add(id);
            }
\end{lstlisting}

\item[4.]

Erstmall die SafeChecker Klasse, das gehört auch zum Teils zur nächsten Aufgabe. 

\begin{lstlisting}[style=java]
package allocationAlgorithm;

import calculationUtil.CUtil;
import process.Process;
import resources.Resources;

public class SafeChecker {

    public static boolean checkIfSafeState(Resources resources, Process[] processes) {
        return checkIfSafeState(processes, getAvailableResources(resources, processes));
    }

    public static boolean checkIfSafeState(Process[] processes, int[] availableResources) {
        int[][] pendingResources = getPendingResources(processes);
        int[][] acquiredResources = getAcquiredResources(processes);

        return checkIfSafeState(pendingResources, acquiredResources, availableResources);
    }

    public static boolean checkIfSafeState(int[][] pendingResources, int[][] acquiredRes, int[] availableResources) {
        assert pendingResources.length == acquiredRes.length;

        boolean[] ready = new boolean[acquiredRes.length];
        boolean changed = true;
        int counter = 0;
        int[] localAvailableRes = availableResources.clone();

        // until there are changes and not all processes are servived
        while (changed && counter < pendingResources.length) {
            changed = false; // a new cycle began, see if there are changes this time

            for (int i = 0; i < pendingResources.length; i++) {
                // check if the current process can be serviced, if so mark it as ready and acquire resources
                if (CUtil.greaterEqual(localAvailableRes, pendingResources[i]) && !ready[i]) {
                    localAvailableRes = CUtil.add(localAvailableRes, acquiredRes[i]);
                    ready[i] = true;
                    changed = true;
                    counter++;
                }
            }
        }

        return counter == pendingResources.length; // are all processes marked as serviced?
    }

    public static int[] getAvailableResources(Resources resources, Process[] processes) {
        int[] totalResources = resources.getTotalRes();
        int[] availableResources = totalResources.clone();

        for (int i = 0; i < processes.length; i++) {
            availableResources = CUtil.sub(availableResources, processes[i].getAcquiredRes());
        }

        return availableResources;
    }

    public static int[][] getPendingResources(Process[] processes) {
        int[][] pendingResources = new int[processes.length][];

        for (int i = 0; i < processes.length; i++) {
            pendingResources[i] = processes[i].getPendingRes().clone();
        }

        return pendingResources;
    }

    public static int[][] getAcquiredResources(Process[] processes) {
        int[][] acquiredResources = new int[processes.length][];

        for (int i = 0; i < processes.length; i++) {
            acquiredResources[i] = processes[i].getAcquiredRes().clone();
        }

        return acquiredResources;
    }

    public static void updateAcquired(int[][] acquiredResources, Process[] processes) {
        for (int i = 0; i < acquiredResources.length; i++) {
            if (processes[i].isFinished()) {
                acquiredResources[i] = CUtil.sub(acquiredResources[i], acquiredResources[i]);
            }
        }
    }
}

\end{lstlisting}

So kann man es danach verwenden (die ganze Klasse ist hier nicht relevant, das ganze Code ist im GitHub):

\begin{lstlisting}[style=java]
return String.format("Safe state: %s", SafeChecker.checkIfSafeState(processes, resources.getFreeRes()));
\end{lstlisting}


\end{enumerate}

\item[e)]
Klasse für Bankieralgorithmus

\begin{lstlisting}[style=java]
package allocationAlgorithm;

import calculationUtil.CUtil;
import process.Process;
import process.Process.AllocationRequest;
import resources.Resources;

public class BankersAlgo implements Algorithm {

    Resources resources;
    Process[] processes;

    int[][] acquiredRes;
    int[][] pendingRes;

    @Override
    public void init(Resources resources, Process[] processes) {
        this.resources = resources;
        this.processes = processes;
        this.pendingRes = SafeChecker.getPendingResources(processes);
        this.acquiredRes = SafeChecker.getAcquiredResources(processes);
    }

    @Override
    public String nextStep(AllocationRequest[] requests) {

        int[] freeRes = resources.getFreeRes();
        SafeChecker.updateAcquired(acquiredRes, processes);

        for (AllocationRequest req : requests) {

            int[] freeResSimulation = freeRes.clone();
            int[][] pendingResSimulation = pendingRes.clone();
            int[][] acquiredResSimulation = acquiredRes.clone();
            int processIdx = this.getProcessIndex(req);

            // simulate resource allocation
            pendingResSimulation[processIdx] = CUtil.sub(pendingResSimulation[processIdx], req.getRequestedRes());
            acquiredResSimulation[processIdx] = CUtil.add(acquiredResSimulation[processIdx], req.getRequestedRes());
            freeResSimulation = CUtil.sub(freeResSimulation, req.getRequestedRes());

            // if state is safe after allocation, grant permission
            if (SafeChecker.checkIfSafeState(pendingResSimulation, acquiredResSimulation, freeResSimulation)) {
                freeRes = CUtil.sub(freeRes, req.getRequestedRes());
                this.pendingRes[processIdx] = CUtil.sub(this.pendingRes[processIdx], req.getRequestedRes());
                this.acquiredRes[processIdx] = CUtil.add(this.acquiredRes[processIdx], req.getRequestedRes());
                req.grant();
            }
        }

        return String.format("safe: %s", SafeChecker.checkIfSafeState(resources, processes));
    }

    public int getProcessIndex(AllocationRequest req) throws IllegalArgumentException {
        for (int i = 0; i < processes.length; i++) {
            if (req.getProcess().getId() == processes[i].getId()) {
                return i;
            }
        }

        throw new IllegalArgumentException("Invalid request!");
    }

}

\end{lstlisting}

\end{enumerate}

% /////////////////////// END DOKUMENT /////////////////////////
\end{document}