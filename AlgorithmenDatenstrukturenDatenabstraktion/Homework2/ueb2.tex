\documentclass[11]{article}
\usepackage[utf8]{inputenc}
\usepackage{amssymb}
\usepackage[parfill]{parskip}
\usepackage[ngerman]{babel}
\usepackage[fleqn]{amsmath}
\usepackage{mathtools}
\usepackage{graphicx}
\usepackage{cancel}
\graphicspath{ {images/} }
\usepackage
[
        a4paper,% other options: a3paper, a5paper, etc
        left=2cm,
        right=2cm,
        top=2cm,
        bottom=3cm
]
{geometry}
{\setlength{\mathindent}{0.5cm}
\allowdisplaybreaks

\begin{document}
\title
{
L\"osungen von \"Ubungsblatt 2 \\
ALP III: Algorithmen, Datenstrukturen und Datenabstraktion \\ (K. Kriegel) \\
\normalsize Tutorium: Wellner, David; Mittwoch, 12:00 - 14:00 \\
\normalsize Tutorium von Boyan: Berendsohn, Benjamin Aram; Dienstag, 16:00 - 18:00
}
\author{Boyan Hristov und Julian Habib}
\date{\today}
\maketitle

\begin{tabular}{|r|r|r|r|r|}
\hline 
Aufgabe 1 & Aufgabe 2 & Aufgabe 3 & Aufgabe 4 & $\qquad \Sigma \qquad$  \\ 
\hline 
/5 & /5 & /5 & /5 & \\ 
\hline 
\end{tabular}

\section*{Aufgabe 1}

\begin{description}
\item[a)]

\begin{align*}
	& f(n) \in O(g(n)) \Rightarrow \exists c_1 > 0 \exists n_{01} \forall n \geq n_{01}: f(n) \leq c_1.g(n) \tag{1} \\
	& h(n) \in \Omega(g(n)) \Rightarrow \exists c_2 > 0 \exists n_{02} \forall n \geq n_{02}: c_2.g(n) \leq h(n) \tag{2} \\ \\
	& \text{Aus (1) und (2) und $c_1, c_2, f(n), g(n), h(n) \geq 0$} \tag{Da es hier um Laufzeitsanalyse geht}\\
	& \Rightarrow \exists c_1, c_2 \forall n_0 = max(n_{01}, n_{02}) \geq n: f(n).c_2.g(n) \leq
		 c_1.g(n).h(n) \\
	& \Rightarrow f(n).c_2 \leq h(n).c_1 \Rightarrow f(n)\frac{c_2}{c_1} \leq h(n) \\ \\
	& \Rightarrow \exists c = \frac{c_2}{c_1} > 0 \exists n_0 \forall n \geq n_0: f(n) \leq c.h(n) 
		\Rightarrow f(n) \in O(h(n)) \\
	& \tag*{$\Box$}
\end{align*}

\item[b)]

\begin{align*}
	& f_1(n) \in O(g_1(n)) \Rightarrow \exists c_1 > 0 \exists n_{01} \forall n \geq n_{01}: f_1(n) \leq c_1.g_1(n) \tag{1} \\
	& f_2(n) \in O(g_2(n)) \Rightarrow \exists c_2 > 0 \exists n_{02} \forall n \geq n_{02}: f_2(n) \leq c_2.g_2(n) \tag{2} \\ \\
	%
	& \text{Aus (1) und (2) } \Rightarrow \\
	& \exists c_1, c_2 > 0 \exists n_0 = max(n_{01}, n_{02}) \forall n \geq n_0: 
		f_1(n) \leq c_1.g_1(n) \land f_2(n) \leq c_2.g_2(n) \\
	& \Rightarrow f_1(n) + f_2(n) \leq c_1.g_1(n) + c_2.g_n(n)
		\leq max(g_1(n), g_2(n)).c_1 + max(g_1(n), g_2(n)).c_2 = \\
	& = max(g_1(n), g_2(n))(c_1 + c_2) \\ \\
	%
	& \Rightarrow \exists c = c_1 + c_2 > 0 \exists n_0 \forall n \geq n_0:
		f_1(n) + f_2(n) \leq max(g_1(n), g_2(n)).c \\
	& \Rightarrow f_1(n) + f_2(n) \in O(max(g_1(n), g_2(n))) \\
	& \tag*{$\Box$}
\end{align*}

\end{description}
\text{} \\ \\ \\

\section*{Aufgabe 2} 

\begin{description}
	\item[a)]
	
	\begin{align*}
		 \sqrt{n}(log(n))^2 < \sqrt{n}\sqrt[2]{n} &= 
		 \tag{Nach 9. Satz von "Werkzeuge und Grundlagen" aus der Vorlesung} \\
		 & =\sqrt{n}^2 = n \\
		 \Rightarrow \sqrt{n}(log(n))^2 \in o(n) \enspace \land \enspace
		 	 & \sqrt{n}(log(n))^2 \in O(n)
	\end{align*}
	
	\item[b)]
	
	\begin{align*}
		& \frac{n^2}{log \enspace n} > \frac{n^2}{n} = n \tag{für $n \geq 2$} \\
		& \Rightarrow \frac{n^2}{log \enspace n} \in \omega(n) \enspace \land \enspace 
			\frac{n^2}{log \enspace n} \in \Omega(n)
	\end{align*}
	
	\item[c)]	
	
	\begin{align*}
		& 4^n = (2^2)^n = 2^{2n} = 2^n.2^n > 2^n \Rightarrow 4^n \notin O(2^n)
	\end{align*}		
	
\end{description}

\section*{Aufgabe 3}

\begin{description}
	\item[a)]
	\begin{align*}
		& f_1(n) = log((n!)^2) = 2.log(n!) \enspace \text{ (1) } \tag{Grund: $log_a(n^b) = b.log_an$} \\
		& log(n!) \in \Theta(n.log(n)) \enspace \text{ (2) }  \tag{5. Regel aus "Grundlagen und Werkzeugen" in der Vorlesung} \\ \\
		& (1) \land (2) \Rightarrow f_1(n) = log((n!)^2) \in \Theta(n.log(n)) \text{ ,da 2 eine Konstante ist}
	\end{align*}
	
	\item[b)]
	\begin{align*}
		& f_2(n) = log((n^2)!) \\
		& \text{Sei } x = n^2 \Rightarrow \tag{Nach 5. Regel aus der Vorlesung} \\
		& \exists c_1, c_2 > 0: c_1.x.log(x) \leq log(x!) \leq c_2.x.log(x) \Rightarrow 
			\tag{Wider x einsetzen} \\ 
		& \Rightarrow \exists c_1, c_2 > 0: c_1.n^2.log(n^2) \leq log((n^2)!) \leq c_1.n^2.log(n^2) \\
		& \Rightarrow \exists c_1, c_2 > 0: c_1.n^2.2.log(n) \leq log((n^2)!) \leq c_1.n^2.2.log(n) \\\\
		& \Rightarrow f_2(n) = log((n^2)) \in \Theta(n^2.log(n)) \tag{Da $c_1.2$, $c_2.2$ Konstanten sind}
	\end{align*}
	
	\item[c)]
	\begin{align*}
		& f_3(n) = log(log(n!)) \Rightarrow \text{Nach 5. Regel} \\
		& \exists c_1, c_2 > 0: log(c_1.n.log(n)) \leq log(log(n!))\leq log(c_2.n.log(n)) \Rightarrow \tag{Nach log(a.b) = log(a) + log(b)} \\
		& \Rightarrow log(c_1) + log(n) + log(log(n) \leq log(log(n!)) \leq log(c_2) + log(n) + log(log(n)) \\ \\
		%
		& \Rightarrow \exists c_1, c_2 > 0: c_1.log(n) \leq log(log(n!)) \leq c_2.log(n) \Rightarrow \tag{Nach Def. von $\Theta$} \\
		& \Rightarrow log(log(n!)) \in \Theta(log(n))
	\end{align*}
	
	\item[d)]
	\begin{align*}
		& f_4(n) = log(n^n) = n.log(n) \tag{da $log_a(n^b) = b.log_an$} \\
		& \Rightarrow f_4(n) \in \Theta(n.log(n))
	\end{align*}
	
	\item[e)]
	\begin{align*}
		f_5(n) = log(4^{n!}) &= n!.log(4) \tag{da $log_a(n^b) = b.log_an$} \\
		&= 2n! \in \Theta(n!)
	\end{align*}
	
\end{description}

\section*{Aufgabe 4}

\begin{description}
	\item[a)] 
	\text{}
	\begin{description}
	\item[3 + 4 + 2*(4 - 2)]
	\text{} \\ \\
	2*(4 - 2) $\equiv$ 2 4 2 - * \\
	4 + 2*(4-2) $\equiv$ 4 2 4 2 - * + \\ \\
	3 + 4 + 2*(4 - 2) $\equiv$ 3 4 2 4 2 - * + +
	
	\item[2 + (3 * 3 - 2 * 4)]
	\text{} \\ \\
	3 * 3 $\equiv$ 3 3 * \\
	2 * 4 $\equiv$ 2 4 * \\
	3 * 3 - 2 * 4 $\equiv$ 3 3 * 2 4 * - \\ \\
	2 + (3 * 3 - 2 * 4) $\equiv$ 2 3 3 * 2 4 * - +
	\end{description}
	
	\item[b)]
	\text{} \\
	
	In dieser Aufgabe haben wir den Stack als ein Array (also so []) vorgestellt. 
	Alle Schritte, in denen nur weitere Operanden geholt werden haben wir 
	übersprüngen.
	
	\begin{description}
	\item[3 4 2 4 2 - * + +]
	\text{} \\ \\
	$[3, 4, 2, 4, 2, -] \rightarrow [3, 4, 2, 2] \rightarrow [3,4,2,2,*]
		\rightarrow [3,4,4] \rightarrow [3,4,4,+] \rightarrow [3,8] \rightarrow [3, 8, +] \rightarrow [11] = 1$
	
	\item[2 3 3 * 2 4 * - +]
	\text{} \\ \\
	$[2, 3, 3, *] \rightarrow [2, 9] \rightarrow [2, 9, 2, 4,*] \rightarrow 
		[2, 9, 8] \rightarrow [2, 9, 8, -] \rightarrow [2, 1] 
			\rightarrow [2, 1, +] \rightarrow [3] = 3$
	
	\end{description}
\end{description}

\end{document}