\documentclass[12pt,a4paper]{article}
%-------------------------------------
\usepackage[utf8]{inputenc}
\usepackage[german]{babel}
\usepackage{amsmath}
\usepackage{amsfonts}
\usepackage{amssymb}
\usepackage{amsthm}
\usepackage{graphicx}
\usepackage{tabularx}
%-------------------------------------------------------------------------
\newcolumntype{L}[1]{>{\raggedright\arraybackslash}p{#1}} % linksbündig mit Breitenangabe
\newcolumntype{C}[1]{>{\centering\arraybackslash}p{#1}} % zentriert mit Breitenangabe
\newcolumntype{R}[1]{>{\raggedleft\arraybackslash}p{#1}} % rechtsbündig mit Breitenangabe
%--------------------------------------------------------------------------
\newtheorem{theorem}{Satz}[section]
\newtheorem{lemma}[theorem]{Lemma}
\theoremstyle{definition}
\newtheorem{definition}[theorem]{Definition}
\newtheorem{bemerkung}{Bemerkung}
%---------------------------------------------------------------------------
\setlength{\tabcolsep}{0cm}
%---------------------------------------------------------------------------
\makeatletter
\def\@maketitle{
\hspace{-0.6cm}\begin{tabular}{L{5cm}C{4.4cm}R{4.3cm}}
                   Proseminar & Theoretische Informatik & WS 17/18\\
\end{tabular}
\hrule
\vspace{0.1cm}
\hrule
\begin{center}%
{\bf \large \@title}
    \\
\end{center}%
\hrule
\vspace{0.1cm}
\hrule
\vspace{0.1cm}
\begin{tabular}{p{5cm}C{4.4cm}R{4.3cm}}
{\bf\@author}
    & \@date & Claudia Dieckmann\\
\end{tabular}
\vspace{1cm} \\
}
\makeatother
%---------------------------------------------------------------------------
%___________________________________________________________________________
%Hier Autor, Titel und Datum eintragen
\author{Boyan Hristov}
\title{Kryptographie}
\date{\today}
%___________________________________________________________________________
%---------------------------------------------------------------------------
%Ab hier beginnt das eigentliche Dokument.
\begin{document}
    \maketitle
    \begin{abstract}
        Wir werden einige Konzepte einführen, die man als die Grundbausteine der Kryptographie betrachtet. Wir werden
        dabei sehen, unter welchen Annahmen sichere Kryptographie existieren kann. Weiter wird der Zusammenhang mit den
        Klassen P und NP erläutert, als auch werden einigen interessanten Schlussfolgerungen gemacht. Am Ende werden
        wir auch die Funktion hinter den berühmten RSA Algorithmus sehen.
    \end{abstract}
    \section{Motivation}

    \subsection{Secret Codes \& One-way Pads}
    Die naive Methode verschlüsselte Konversation zu haben ist, wenn Sender und Empfänger der gleiche Schlüssel benutzen,
    um Nachrichten zu verschlüsseln und entschlüsseln. Dabei existiert das Problem, dass Sender und Empfänger sich auf
    einen Schlüssel einigen müssen. Das ist aber schwierig, da bevor sie das gemacht haben, können sie auch keine
    sicheren Nachrichten zwischen einander senden. Weiter ist das keine gute Lösung, da für je Gesprächspaar ein
    Schlüssel existieren muss, d.h. wenn jemand mit vielen Leuten verschlüsselte Nachrichten austauschen will, muss diese
    Person auch viele Schlüssel speichern. \\

    Laut Theorie ist ein Schlüssel nur dann sicher, wenn er mindestens so lang ist wie alle Nachrichten, die damit
    verschlüsselt werden sollen. Deswegen sind die sogenannte ``One-way Pads`` entstanden. Dabei wird je Nachricht mit
    einem Teil des Schlüssels verschlüsselt und dieses Teil wird nachher gelöscht. So garantiert man, dass ein
    Schlüssel durch ``Brute-Force`` schwierig zu finden ist. Wie wir aber später sehen werden, reicht es auch wenn der
    Schlüssel relativ lang ist, solange es keine andere Methode außer Brute-Force gibt, den Schlüssel leichter zu
    finden. \\

    So kommen wir zu den Konzepten der sicheren Hashfunktionen und der asymmetrischen Verschlüsselung,
    wobei eine Nachricht mit einem Schlüssel verschlüsselt wird und mit einem anderen entschlüsselt.

    \subsection{Ziele der Public-Key Kryptosysteme}
    Hier werden wir einige Notationen einführen.

    \begin{itemize}
        \item $M \in \Sigma^*$ - eine Nachricht
        \item $E: \Sigma^* \longrightarrow \Sigma^*$ - Funktion, die eine Nachricht verschlüsselt
        \item $D: \Sigma^* \longrightarrow \Sigma^*$ - Funktion, die eine Nachricht entschlüsselt
    \end{itemize}

    Jedes ``Public-Key`` Kryptosystem muss die folgenden Eigenschaften haben. Wir werden später sehen, was für
    Annahmen gemacht werden müssen, um diese Eigenschaften zu sichern.

    \begin{enumerate}
        \item E und D sind Polynomialzeit berechenbar
        \item Man muss die originelle von verschlüsselten Nachrichten generieren können und zwar mit dem Gegenschlüssel. \\
        Formal: $D(E(M)) = M$
        \item Mann muss eine Nachricht erstmal entschlüsseln und dann verschlüsseln können, wobei wieder die originelle
        Nachricht entsteht. \\
        Formal: $E(D(M)) = M$
        \item Wenn ein Nutzer E veröffentlicht, veröffentlich er damit nicht D und auch kein Weg, D leichter
        (in polynomialer Zeit) zu berechnen
    \end{enumerate}

    \section{Erinnerung}

    Die meisten Leser können dieses Teil überspringen, es ist aber wichtig zu verstehen, was links- und
    rechtsinvertierbarkeit bedeuten und welche Folgen diese für die Kryptographie haben.

    \subsection{Injektivität}

    \begin{definition}
        Eine Funktion $f: X \longrightarrow Y$ ist injektiv \\
        $\Leftrightarrow (f(a) = f(b) \Leftrightarrow a = b)$
    \end{definition}

    \subsection{Surjektivität}

    \begin{definition}
        Eine Funktion $f: X \longrightarrow Y$ ist surjektiv \\
        $\Leftrightarrow \forall y \in Y \exists x \in X: f(x) = y$
    \end{definition}

    \subsection{Linksinvertierbarkeit}
    Hier ist I die Identitätsfunktion und $\circ$ die Verkettung von Funktionen.
    \begin{definition}
        Eine Funktion $f: X \longrightarrow Y$ ist linksinvertierbar \\
        $\Leftrightarrow \exists g: Y \longrightarrow X: g \circ f = I_x$ \\
        $\Leftrightarrow f$ ist injektiv
    \end{definition}

    \subsection{Rechtsinvertierbarkeit}
    Hier ist I die Identitätsfunktion und $\circ$ die Verkettung von Funktionen.
    \begin{definition}
        Eine Funktion $f: X \longrightarrow Y$ ist rechtsinvertierbar \\
        $\Leftrightarrow \exists g: Y \longrightarrow X: f \circ g = I_x$ \\
        $\Leftrightarrow f$ ist surjektiv
    \end{definition}


    \section{One way functions (Einwegfunktionen)}

    In diesem Abschnitt werden die Einwegfunktionen in starke und schwache unterteilt. Es wird auch gezeigt, dass die
    existenz von schwachen Einwegfunktionen diese von starken impliziert. Weiter werden wir sehen, dass eine
    längenerhaltende Funktion aus eine nicht längenerhaltende konstruiert werden kann. Die starke Einwegfunktionen
    sind ein grundlegender Baustein für die Kryptographie, Ziel ist aber hier auch alle Annahmen zu verstehen die dazu
    führen. \\
    Da die meisten Definitionen auf probabilistische Algorithmen sich basieren, werden wir erstmal definieren, was
    vernachlässige Erfolgswahrscheinlichkeit bedeutet, um besser die folgenden Konzepte erklären zu können.

    \subsection{Vernachlässige Erfolgswahrscheinlichkeit}

    \begin{definition}
        Vernachlässige Erfolgswahrscheinlichkeit
        \begin{itemize}
            \item Funktion abhängig von der Eingabegroße
            \item Die Erfolgswahrscheinlichkeit eines Algorithmus ist genau dann vernachlässig, wenn es eine
                asymptotische obere Schranke gibt, die polynomial ist
            \item D.h, auch wenn man polynomial lange den Algorithmus wiederholt, ist die Erfolgswahrscheinlichkeit
                immer noch vernachlässig
            \item Damit ist vernachlässige Erfolgswahrscheinlichkeit $\equiv$ nicht Polynomialzeit berechenbar
                mit einer nicht vernachlässigen Erfolgswahrscheinlichkeit
        \end{itemize}
    \end{definition}

    \subsection{Starke und Schwache Einwegfunktionen}

    \begin{definition}
        Eine Funktion $f: {0, 1}^* \longrightarrow {0, 1}^*$ ist eine starke one-way Funtkion $\Leftrightarrow$
        \begin{enumerate}
            \item f ist Polynomialzeit berechenbar
            \item Für jeden in polynomialer Zeit berechenbaren Algorithmus A, für jedes Polynom p und für große n gilt: \\
            $Pr(A(f(w), 1^n) \in f^{-1}(f(w))) \leq \frac{1}{p(n)}$
        \end{enumerate}

        Damit hat A per Definition eine vernachlässigbare Erfolgswahrscheinlichkeit. $1^n$ wird als Parameter übergeben um zu
        garantieren, dass die inverse Funktion echt Polynomialzeit unberechenbar ist. Wenn man das weglässt, könnte es sein,
        dass die Eingabegröße von f exponentiell größer ist als die Ausgabegröße und damit f nur deswegen nicht invertierbar
        ist, weil kein Algorithmus in polynomialer Zeit das Ergebnis auf dem Band schreiben kann.
    \end{definition}


    % define weak one way function

    \begin{definition}
        Eine Funktion $f: {0, 1}^* \longrightarrow {0, 1}^*$ ist eine schwache one-way Funtkion $\Leftrightarrow$
        \begin{enumerate}
            \item f ist Polynomialzeit berechenbar
            \item Es existiert ein Polynom p, so dass für alle polynomialzeit berechenbare Algorithmen A und große n: \\
            $Pr(A(f(w), 1^n) \notin f^{-1}f(w)) > \frac{1}{p(n)}$
        \end{enumerate}

    \end{definition}

    Merke, dass f invertierbar sein kann, nur nicht in polynomialer Zeit. Weiter steht in der Definition nicht fest,
    ob f injektiv oder surjektiv ist, da hier f nicht unbedingt in mathematischem Sinne invertierbar ist. D.h. auch
    $f(x) = 2$ ist laut dieser Definition keine Einwegfunktion, da $g(x) = 1$ schon als zu f inverse Funktion gilt.
    Surjektivität werden wir für ``Public-Key`` Kryptosysteme auf jeden Fall brauchen. Ohne Injektivität kann man schon
    eine Hashingfunktion erstellen (es existieren viele solche), aber keine mathematisch sichere, da mehrere
    unterschiedliche $w \in \Sigma^*$ dieselbe Hashwerte bekommen werden.

    Hoffnung für sichere Hashfunktionen gibt es erst mit den Einwegpermutationen, die wir bald sehen werden. Bevor
    müssen wir aber auch die längenerhaltende Funktionen einführen, die in der Praxis ganz hilfreich sind.

    % end of weak one way function definition

    \begin{theorem}
        Es existiert eine schwache Einwegfunktion $\Leftrightarrow$ eine starke Einwegfunktion existiert.
    \end{theorem}
    \begin{proof}
        Der Beweis dafür ist ziemlich lang und kompliziert, deswegen werden wir hier nur die Schema erläutern. In
        dem Buch von Oded Goldreich ``Foundations of Cryptography - Basic Tools`` kann man im Abschnitt 2.3.2 den
        vollständigen Beweis finden.

        Die Idee ist aus einer schwachen Einwegfunktion f, eine starke Einwegfunktion g folgendermaßen zu konstruieren.

        $g(x_1, ..., x_{t(n)}) = f(x_1),...,f(x_{t(n)})$, wobei $|x_1| = |x_{t(n)}|$ und $t(n) = n . p(n)$. Also
        $n^2p(n)$-bit lange Eingabe wird in n Blöcke je mit Länge $t(n)$. \\ \\

        Der Rest des Beweises nimmt an, dass g keine starke Einwegfunktion ist und konstruiert daraus ein Algorithmus,
        der f in polynomialer Zeit mit vernachlässigbare Fehlerwahrscheinlichkeit invertiert, woraus der Widerspruch
        entsteht. \\ \\

        Der Beweis ist so konstruiert, da man nicht sicher sein kann, das ein solcher Algorithmus jeden Block
        einzeln invertieren wird, was in einer maximalen Erfolgswahrscheinlichkeit von $(1 - \frac{1}{p(n)})^{np(n)}$
        resultieren wurden. Das zeigt, dass Verstärkung von der Komplexität einer Aufgabe deutlich schwieriger ist
        als probabilistische Verstärkung.
    \end{proof}

    \subsection{Längenerhaltende Funktionen}

    \begin{definition}
        Eine Funktion $f: \Sigma^* \longrightarrow \Sigma^*$ ist
        längenerhaltend $\Leftrightarrow \forall w \in \Sigma^{*}: |f(w)| = |w|$
    \end{definition}
    \begin{theorem}
        Wenn eine Einwegfunktion existiert, dann existiert auch eine längenerhaltende Einwegfunktion
    \end{theorem}
    \begin{proof}
        Sei $f: \Sigma^* \longrightarrow \Sigma^*$ eine starke Einwegfunktion.
        Da f Polynomialzeit berechenbar ist $\Rightarrow \forall x \in \Sigma^*: |f(x)| \leq p(|x|)$, d.h. die
        Länge von jeder Abbildung von f ist polynomial in der Länge begrenzt. \\
        Wir konstruieren $g: \Sigma^* \longrightarrow \Sigma^*$ und $h: \Sigma^* \longrightarrow \Sigma^*$. \\
        $g(x) =^{def} f(x)10^{p(|x|) - |f(x)|}$. \\
        Damit haben wir festgelegt, dass jede Abbildung von g die maximale
        Länge von allen Abbildungen von f haben wird. \\
        $h(x'x'') =^{def} g(x')$, wobei $|x'x''| = p(|x'|) + 1$. \\
        Damit konstruieren wir h so, dass je Eingabe die gleiche
        Länge hat. Die Eingabemenge wird wieder künstlich gefüllt, die letzte Zeichen werden aber ignoriert. \\ \\
%
        Damit ist trivialerweise h eine längenerhaltende Funktion. Jetzt müssen wir nur zeigen, dass diese auch
        eine starke Einwegfunktion ist. Der Beweis per Widerspruch folgt.\\ \\
        %
        Sei B ein Algorithmus, dass g in polynomialer Zeit mit unvernachlässige Erfolgswahrscheinlichkeit invertiert.
        Wir konstruieren den Algorithmus A, der f in polynomialer Zeit invertiert. \\
        Bei Eingabe $(y, 1^n)$ hällt dann A mit Ausgabe $B(y10^{p(n) - |y|}, 1^{p(n) + 1})$. \\
        Damit haben wir gezeigt, dass $B \geq_p A$. Widerspruch, da A in polynomialer Zeit nur mit vernachlässige
        Erfolgswahrscheinlichkeit terminiert.

        Analog für h.
    \end{proof}

    \section{One-way Permutationen}

    Hier wurde die Definition aus ``Introduction to the Theory of Computation`` von Michael Sipser benutzt, jedoch
    etwas angepasst, damit die Konvention weiter ähnlich bleibt.

    \begin{definition}
        Eine Funktion $f: \Sigma^* \longrightarrow \Sigma^*$ ist eine Einwegpermutation $\Leftrightarrow$
        \begin{enumerate}
            \item f ist eine längenerhaltende Permutation
            \item f ist Polynomialzeit berechenbar
            \item Für alle probabilistische Algorithmen A, $\forall k \in N^+$ und für große n, wählen wir ein Wort w
            mit $|w| = n$ und führen A mit Eingabe w aus \\
            $Pr_{A, w}(A(f(w)) = w) \leq n^{-k}$
        \end{enumerate}
    \end{definition}

    Hier wurde auch die Definition einer starken Einwegpermutation benutzt. $Pr_{A, w}$ notiert die Wahrscheinlichkeit
    über die beliebige Wahl von w als auch die Münzwürfe von A.
    Wie wurde diese einer schwachen aussehen?

    \subsection{Bedeutung}
    Die starke längenerhaltende Einwegpermutationen sind der Grundbaustein der Kryptographie. Wenn solche existieren,
    kann man sichere Hashingalgorithmen erstellen. Mit solchen werden heutzutage z.B. die Passwörter hashiert, bevor
    diese in einem Datenbanksystem gespeichert werden. So kann jedes mal das System das von Nutzer eingegebenes Passwort
    hashieren und mit dem gespeicherten vergleichen. Wenn jemand die Passwörter aus dem Datenbank sehen kann, kann
    dieser nie die originelle Klartextpasswörter von diesen erstellen.

    \section{Trapdoor Functions (Falltürfunktionen)}

    \begin{definition}
        Eine Funktion $f: \Sigma^* \longrightarrow \Sigma^*$ ist eine indexierte Funktion, für die ein
        probabilistischer Polynomialzeit Algorithmus A
        existiert, und eine zu f inverse Funktion $h: \Sigma^* \longrightarrow \Sigma^*$.
        Hier ist A der Algorithmus, der f, h, i und t
        liefert. $A'$ ist der Algorithmus, der versucht, f zu invertieren.
        Dabei müssen f, g und h die folgenden Eigenschaften haben.
        \begin{enumerate}
            \item f und h sind Polynomialzeit berechenbar
            \item Für alle probabilistische Algorithmen $A'$, $\forall k \in N^+$ und für große n, wählen wir eine
            beliebige Rückgabe $(i, t)$ von A bei Eingabe $(w \in \Sigma^n, 1^n)$ \\
            $Pr(A'(i, f_i(w)) = y | f_i(y) = f_i(w)) \leq n^{-k}$
            \item $\forall n, \forall w \in \Sigma^n$, für je Rückgabepaar (i, t) von A, dass mit nichtvernachlässige
            Wahrscheinlichkeit vorkommt, gilt: \\
            $h(t, f_i(w)) = y$, wobei $f_i(y) = f_i(w)$
        \end{enumerate}
    \end{definition}

    Hier ist t das Gehemnis, womit f leicht invertierbar ist. Die zweite Eigenschaft sagt, dass f ohne t schwer
    invertierbar ist und die dritte, dass f mit Hilfe von t leicht invertierbar ist und h die entsprechende
    inverse Funktion ist.

    \subsection{Bedeutung}

    Das ist schon alles, was wir brauchen, um ein ``Public-Key`` Kryptosystem zu konstruieren. Die Funktion $f_i$
    können wir veröffentlichen, das wird unser öffentliches Schlüssel sein. Die Funktion h bzw. das Gehemnis t werden
    wir für uns behalten.

    \section{Beispiel}

    \subsection{Konstruktion der Trapdoor Funktion hinter RSA}

    \begin{enumerate}
        \item Wähle 2 (echt) große Primzahlen p und q
        \item Berechne $N = pq$
        \item Berechne $\phi(N) = \phi(pq) = \phi(p) \phi(q) = (p-1)(q-1)$
        \item Wähle $\phi(N) < e < N$ zufällig, so dass es semiprim zu $\phi(N)$ ist. Je Primzahl $> \phi(N)$ ist möglich.
        \item Berechne d, so dass $ed \equiv 1 mod \phi(N)$. Das ist immer möglich, da alle Zahlen zwischen 1 und
            $\phi(N)$ formen eine Gruppe, die abgeschlossen bezüglich Multiplikation Modulo $\phi(N)$ ist.
            Dazu bietet sich das euclidische Algorithmus für die Suche nach $GGT(\phi(N), e)$.
    \end{enumerate}

    \subsection{Anwendung von RSA}

    Aus dem oben beschriebenen Algorithmus werden wir e und N veröffentlichen und d für uns behalten. Damit gilt:

    \begin{itemize}
        \item $E(M) = M^e mod N$
        \item $D(M) = M^d mod N$
        \item Damit $D(E(M)) = (M^e mod N)^d mod N = M^{ed} mod N = M$
        \item Und $E(D(M)) = M^{ed} mod N = M$ analog
    \end{itemize}

    So können wir Nachrichten beides verschlüsseln (mit dem öffentlichen Schlüssel von jemandem, nur diese Person
    kann das öffnen), oder signieren, wobei wir eine Nachricht mit unserem privaten Schlüssel verschlüsseln. Wenn die
    Nachricht sich mit unserem öffentlichen Schlüssel sich entschlüsseln lässt, dann wurde diese sicherlich mit unserem
    privaten Schlüssel verschlüsselt und wurde damit sicherlich von uns geschrieben.

    \section{Konsequenzen von der Existenz der Einwegfunktionen}

    Obwohl es eine starke Vermutung gibt, dass $P \not = NP$, macht es immer Sinn nachzudenken, was für Bedeutung
    $P = NP$ für die Kryptographie hat, bzw. wie sicher wir sind, wenn $P \not = NP$ bewiesen wird.

    \begin{theorem}
        $P = NP \Rightarrow$ es existieren keine Einwegfunktionen
    \end{theorem}

    \begin{proof}
        Sei $f: \Sigma^* \longrightarrow \Sigma^*$, $\Sigma = {0, 1}$, eine Einwegfunktion. \\
        Sei L eine Sprache von Paaren, so dass $(x^*, y) \in L \Leftrightarrow \exists x: f(x) = y$ und $x^*$ ist ein
        Präfix von x. \\
        $L \in NP$, da bei ein Eingabe $(x^*, y)$ und Zeuge $(x, y)$ können wir in polynomialer Zeit überprüfen,
        ob $f(x) = y$ (da f Polynomialzeit berechenbar ist, nach der Definition einer Einwegfunktion) und ob $x^*$ ein
        Präfix von x ist. \\ \\

        Da aber nach Annahme $P = NP \Rightarrow L \in P$. Sei A das Algorithmus, dass L in polynomialer Zeit
        entscheidet. Damit können wir L benutzen um f in polynomialer
        zu invertieren, also um aus je Ausgabe y das entsprechende Eingabewert x zu generieren. \\
        Wir fangen mit $(\epsilon, y)$. Dann raten wir das nächste Zeichen von x und benutzen A um zu überprüfen,
        ob wir richtig geraten haben. Da A in polynomialer Zeit läuft, haben wir am Ende eine Laufzeit von O($|y|p$),
        wobei p ein Polynom ist. Damit haben wir in polynomialer Zeit f invertiert, deswegen ist per Definition
        f keine Einwegfunktion. Widerspruch!
    \end{proof}

    \section{Literatur}

    \begin{itemize}
        \item Introduction To The Theory Of Computation, Michael Sipser
        \item Foundations of Cryptography - Basic Tools, Oded Goldreich
        \item https://crypto.stackexchange.com/questions/39878/how-to-show-that-a-one-way-function-proves-that-p-%E2%89%A0-np
        \begin{itemize}
            \item Beweis für $P = NP \Rightarrow$ Einwegfunktionen existieren nicht
        \end{itemize}
        \item Construct a length preserving one way function, Stackexchange
        \begin{itemize}
            \item https://cs.stackexchange.com/questions/10639/length-preserving-one-way-functions
            \item Oben wurde den Beweis von Oded Goldreich benutzt, das war aber für das Verständniss hilfreich
        \end{itemize}
        \item A Method for Obtaining Digital Signatures and Public-Key Cryptosystems,
            R.L. Rivest, A. Shamir, und L. Adleman
        \begin{itemize}
            \item Wie die Namen der Authoren schon zeigt, ist das das Paper zu RSA
        \end{itemize}
        \item https://en.wikipedia.org/wiki/One-way\_function
    \end{itemize}

\end{document}
