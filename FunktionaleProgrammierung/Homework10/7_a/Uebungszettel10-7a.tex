\documentclass[11]{article}
\usepackage[utf8]{inputenc}
\usepackage{amssymb}
\usepackage[parfill]{parskip}
\usepackage[ngerman]{babel}
\usepackage[fleqn]{amsmath}
\usepackage{mathtools}
\usepackage{graphicx}
\graphicspath{ {images/} }
\usepackage
[
        a4paper,% other options: a3paper, a5paper, etc
        left=2cm,
        right=2cm,
        top=2cm,
        bottom=3cm
]
{geometry}
{\setlength{\mathindent}{0.5cm}
\allowdisplaybreaks

\begin{document}
\title
{
L\"osungen von \"Ubungsblatt 9 \\
Funktionale Programmierung (Prof. Dr. Margarita Esponda) \\
\normalsize Tutorium: Zachrau, Alexande; Dienstag; 12:00 - 14:00
}
\author{Boyan Hristov und Luis Herrmann}
\date{\today}
\maketitle

\section*{7a}

\begin{align*}
& S(KK)I \equiv \tag*{S Anwenden} \\
& \equiv (\lambda xyz.xz(yz)(KK)I) \equiv \tag*{Vereinfachen} \\
& \equiv (\lambda yz.(KK)z(yz))I \equiv \tag*{Vereinfachen} \\
& \equiv \lambda z.(\lambda b.K)z(Iz) \equiv \tag*{Vereinfachen} \\
& \equiv \lambda z.K(Iz) \equiv \tag*{K Anwenden} \\
& \equiv \lambda z.(\lambda ab.a)(Iz) \equiv \tag*{Vereinfachen} \\
& \equiv \lambda z. \lambda b.Iz \equiv \tag*{Vereinfachen} \\
& \equiv \lambda zb.(Iz) \equiv \tag*{I Anwenden} \\
& \equiv \lambda zb.((\lambda x.x)z) \equiv \tag*{Vereinfachen} \\
& \equiv \lambda zb.z \equiv \tag*{K Anwenden} \\
& \equiv K
\end{align*}

\end{document}