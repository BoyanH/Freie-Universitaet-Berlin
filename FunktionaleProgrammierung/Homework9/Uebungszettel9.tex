\documentclass[11]{article}
\usepackage[utf8]{inputenc}
\usepackage{amssymb}
\usepackage[parfill]{parskip}
\usepackage[ngerman]{babel}
\usepackage[fleqn]{amsmath}
\usepackage{mathtools}
\usepackage{graphicx}
\graphicspath{ {images/} }
\usepackage
[
        a4paper,% other options: a3paper, a5paper, etc
        left=2cm,
        right=2cm,
        top=2cm,
        bottom=3cm
]
{geometry}
{\setlength{\mathindent}{0.5cm}
\allowdisplaybreaks

\begin{document}
\title
{
L\"osungen von \"Ubungsblatt 9 \\
Funktionale Programmierung (Prof. Dr. Margarita Esponda) \\
\normalsize Tutorium: Zachrau, Alexande; Dienstag; 12:00 - 14:00
}
\author{Boyan Hristov und Luis Herrmann}
\date{\today}
\maketitle

\begin{tabular}{|r|r|r|r|r|r|r|r}
\hline 
Aufgabe 1 & Aufgabe 2 & Aufgabe 3 & Aufgabe 4 & Aufgabe 5 & Aufgabe 6 & Aufgabe 7 \\ 
\hline 
/6 & /4 & /3 & /3 & /6 & /4 & /6\\ 
\hline 
\end{tabular}
%
\section*{Aufgabe 1}
\begin{description}
\item[A)] \textbf{}
\begin{description}
\item[1)] \textbf{Syntaktisch inkorrekt} - Falsche Klammerung zwischen Argumententeil und Rumpf der Funktion 
\item[2)] \textbf{Syntaktisch inkorrekt} - Die Funktion hat kein Rumpf, nur Argumententeil
\item[3)] \textbf{Syntaktisch korrekt} - es gibt aber ein Namenskonflikt zwischen zwei gebundene "a" Variablen
\item[4)] \textbf{Syntaktisch inkorrekt} - Argumente ohne $\lambda$ - Zeichen
\item[5)] \textbf{Syntaktisch korrekt}
\item[6)] \textbf{Syntaktisch korrekt}
\end{description}
%
\item[B)] \textbf{}
\begin{description}
\item[3)] gebunden: a, a; \; frei: y,z,w;
\item[5)] gebunden: a, z, y; \; frei: x, b, c;
\item[6)] \text{ } In $\lambda a.abx: \qquad$ gebunden: a; \; frei: b,x; \\
In $\lambda xyz.x(yzw): \qquad$ gebunden: x,y,z; \; frei: w;
\end{description}
\end{description}
%
\section*{Aufgabe 2}
\begin{description}
\item[1)] \textbf{}
\begin{align*}
& (\lambda xy.x(\lambda abc.b(abc))y) (\lambda sz.z) (\lambda sz.s(z)) \equiv \\
\equiv & (\lambda sz.z) (\lambda abc.b(abc))(\lambda sz.s(z)) \equiv \\
\equiv & \lambda sz.s(z) \equiv 1
\end{align*}
\item[2)] \textbf{}
\begin{align*}
& (\lambda xy.xy(\lambda ab.b))(\lambda ab.a) (\lambda ab.b)xy \equiv \\
\equiv & ((\lambda ab.a)(\lambda ab.b)(\lambda ab.b))xy \equiv \\
\equiv & (\lambda ab.b) x y \equiv y
\end{align*}
\end{description}
%
\section*{Aufgabe 3}
\begin{align*}
& \text{Identitätsfunktion} \equiv \lambda x.x \\ \\
& \land T \equiv (\lambda xy.xyF)(\lambda ab.a) \equiv \\
& \equiv \lambda y.(\lambda ab.a)yF \equiv \lambda y.y \equiv \lambda x.x \\ \\
& \Rightarrow \land T \equiv \text{Identitätsfunktion}
\end{align*}
%
\section*{Aufgabe 4}
\begin{align*}
& F \neg \equiv (\lambda ab.b)(\lambda x.xFT) \equiv \\
& \equiv (\lambda ab.b)(\lambda x.x(\lambda ab.b)(\lambda ab.a)) \equiv \\
& \equiv (\lambda az.z)(\lambda x.x(\lambda ab.b)(\lambda ab.a)) \equiv \\
& \equiv (\lambda z.z) \equiv \lambda x.x \\ \\
& \Rightarrow \text{Identitätsfunktion} \equiv F \neg
\end{align*}
%
\section*{Aufgabe 5}
\begin{description}
\item[A)]
\begin{align*}
& (>) \equiv \neg (<=) \\ \\
& (>) \equiv G \\
& G \equiv \lambda xy.(\lambda a.aFT)(Z (yPx)) \\ \\
& (\lambda a.aFT) \rightarrow \neg \text{Funktion aus der Vorlesung} \\
& (Z (yPx)) \rightarrow (<=) \text{Funktion aus der Vorlesung} \\
& Z \rightarrow istNull \text{Funktion aus der Vorlesung} \\
& P \rightarrow \text{Vorgängerfunktion aus der Vorlesung} \\ \\
& Z \equiv \lambda x.xF\neg F \\ \\
& P \equiv (\lambda n.nH(\lambda z.z00)F) \\
& H \rightarrow \text{erzeugt Nachgänger Tupel }( (n, n-1) \rightarrow (n+1, n) )
\end{align*}
\item[B)]
\begin{align*}
& (<) \equiv \neg (>=) \equiv L \\
& L \equiv \lambda xy.(\lambda b.bFT)(\lambda xy.Z(xPy)) 
\end{align*}
\item[C)]
\begin{align*}
& (\not=) \equiv \neg (=) \equiv U \\ \\
& U \equiv \lambda xy.(\lambda b.bFT)(\land (Z(xPy)) (Z (yPx))) \equiv \\
& \equiv \lambda xy.(\lambda b.bFT)( (\lambda pq.pqF ) (Z(xPy)) (Z (yPx)))
\end{align*}
\end{description}
%
\section*{Aufgabe 6}
In der Vorlesung: $\land \equiv \lambda xy.xyF$ \\
Unser Vorschlag: $a \land b \equiv \neg (\neg a \lor \neg b) \Rightarrow$

\begin{align*}
& \land \equiv A \equiv \lambda xy.(\lambda a.aFT)( (\lambda de.dTe) ((\lambda b.bFT)x) ((\lambda c.cFT)y) ) \equiv \\
& \equiv \lambda xy.(\lambda a.aFT)((\lambda de.dTe) (xFT) (yFT)) \equiv \\
& \equiv \lambda xy.((\lambda de.dTe) (xFT) (yFT))FT \equiv \\
& \equiv \lambda xy.((xFT)T(yFT))FT
\end{align*}
Beweis:
\begin{align*}
& AFF \equiv (\lambda xy.((xFT)T(yFT))FT)FF \equiv ((FFT)T(FFT))FT \equiv (TTT)FT \equiv TFT \equiv F \\
& AFT \equiv (\lambda xy.((xFT)T(yFT))FT)FT \equiv ((FFT)T(TFT))FT \equiv (TTF)FT \equiv TFT \equiv F \\
& ATF \equiv (\lambda xy.((xFT)T(yFT))FT)FT \equiv ((TFT)T(FFT))FT \equiv (FTT)FT \equiv TFT \equiv F \\
& ATT \equiv (\lambda xy.((xFT)T(yFT))FT)FT \equiv ((TFT)T(TFT))FT \equiv (FTF)FT \equiv FFT \equiv T
\end{align*}
%
\section*{Aufgabe 7}
\begin{align*}
& g 0 = 1 \\
& g n = 1+ (g(n-1))*3 \\ \\
& \text{Multiplikation} \equiv M \equiv (\lambda xya.x(ya)) \\ \\
G & \equiv \lambda rn.Zn1(1S (M (r(Pn))3)) \equiv \\
& \equiv \lambda rn.Zn(\lambda sz.s(z))((\lambda ab.a(b))S (M (r(Pn))(\lambda pq.p(p(p(q))) ) )) \\ \\
& \text{Die Zn vergleicht immer ob n=0 ist, wenn das stimmt gibt es 1 zurück. Wir müssen nur noch unsere} \\
& \text{rekursion dazu addieren. Das machen wir wie in der Vorlesung mit dem Y Macro. Damit ist } g \equiv YG
\end{align*}

\end{document}