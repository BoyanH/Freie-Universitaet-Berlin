\documentclass[11, 12pt]{article}
\usepackage[utf8]{inputenc}
\usepackage[T1]{fontenc}
\usepackage{amssymb}
\usepackage[parfill]{parskip}
\usepackage[ngerman]{babel}
\usepackage[fleqn]{amsmath}
\usepackage{mathtools}
\usepackage{enumerate}
\usepackage{graphicx}
\usepackage{color}
\usepackage
[
        a4paper,% other options: a3paper, a5paper, etc
        left=2cm,
        right=2cm,
        top=2cm,
        bottom=3cm
]
{geometry}
{\setlength{\mathindent}{0.5cm}
\allowdisplaybreaks

%Definitions:
\definecolor{bg}{RGB}{230,230,230}
\newcommand{\haskell}[1]{\mintinline{haskell}{#1}}

\begin{document}
\title
{
Lösungen von Übungsblatt 9 \\
Funktionale Programmierung (Prof. Dr. Margarita Esponda) \\
\normalsize Tutorium: Zachrau, Alexander; Dienstag; 12:00 - 14:00
}
\author{Boyan Hristov und Luis Herrmann}
\date{\today}
\maketitle

\begin{tabular}{|r|r|r|r|r|r|r|r}
\hline 
Aufgabe 1 & Aufgabe 2 & Aufgabe 3 & Aufgabe 4 & Aufgabe 5 & Aufgabe 6 & Aufgabe 7 \\ 
\hline 
/6 & /4 & /3 & /3 & /6 & /4 & /6\\ 
\hline 
\end{tabular}
%

div x y = div` y x

div`:

h = $\pi^3_1$ $\circ$ [$\pi^3_2$, sub $\circ$ [$\pi^3_3$, $\pi^3_2$]]
g = Z

\section*{Aufgabe 3}
\begin{description}
\item[A)]
Motivation: Eine Lösung wäre, wie in der Vorlesung, einfach beide Werte mit einander zu multiplizieren, da mul eine primitiv rekursive Funktion ist. Wir haben And auch mit zwei Funktionen g und h definiert, wobei g konstante 0 zurückgibt, wenn das erste Argument eine Null ist, und sonst das zweite Argument.

g = $C^1_0$ \\
h = $\pi^2_2$ \\

\item[B)]
Motivation: Wenn zwei Zahlen gleich sind, dann solle beide Min und Max Funktion gleiche Werte zurückgeben. Wir geben bei Equal deswegen (min+1) - max. Wir nutzen dabei das Eigenschaft, dass 0-x=0 für jedes x definiert ist.

equal = sub $\circ$ [S$\circ$[min $\circ$ [$\pi^2_1, \pi^2_2$]], max $\circ$ [$\pi^2_1, \pi^2_2$]] 

\end{description}

\section*{Aufgabe 4}
\begin{description}
\item[A)]
f = add $\circ$ [$\pi^3_1$, div $\circ$ [mul $\circ$ [add $\circ$ [$\pi^3_1$, $\pi^3_3$], [$\pi^3_1$, $\pi^3_3$]], S $\circ$[$C^3_2$] ]]
g = Z
\item[B)]
p = sub $\circ$ [pow $\circ$ [$C^1_2$, $\pi^1_1$], $C^1_1$]
\item[C)]
abst = add $\circ$ [sub $\circ$ [$\pi^2_1$, $\pi^2_2$], sub $\circ$ [$\pi^2_2$, $\pi^2_1$]]
\item[D)]
g = $C^1_1$
h = add $\circ$ [$\pi^2_1$ $\circ$ [$\pi^2_2$], $\pi^2_2$] 
\end{description}

\end{document}