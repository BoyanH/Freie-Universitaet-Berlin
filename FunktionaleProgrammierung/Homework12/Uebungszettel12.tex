\documentclass[11]{article}
\usepackage[utf8]{inputenc}
\usepackage{amssymb}
\usepackage[parfill]{parskip}
\usepackage[ngerman]{babel}
\usepackage[fleqn]{amsmath}
\usepackage{mathtools}
\usepackage{graphicx}
\graphicspath{ {images/} }
\usepackage
[
        a4paper,% other options: a3paper, a5paper, etc
        left=2cm,
        right=2cm,
        top=2cm,
        bottom=3cm
]
{geometry}
{\setlength{\mathindent}{0.5cm}
\allowdisplaybreaks

\begin{document}
\title
{
L\"osungen von \"Ubungsblatt 12 \\
Funktionale Programmierung (Prof. Dr. Margarita Esponda) \\
\normalsize Tutorium: Zachrau, Alexande; Dienstag; 12:00 - 14:00
}
\author{Boyan Hristov und Luis Herrmann}
\date{\today}
\maketitle

\begin{tabular}{|r|r|r|r|r}
\hline 
Aufgabe 1 & Aufgabe 2 & Aufgabe 3 & Aufgabe 4 \\ 
\hline 
/6 & /4 & /6 & /8\\ 
\hline 
\end{tabular}
%
\section*{Aufgabe 2}
Zu zeigen: $length \; (powset \; xs) \equiv 2^{length \; xs}$ \\ \\
Beweis per strukturelle Induktion \\ \\
%
\textbf{IA: } \\
\begin{equation*}
\begin{aligned}
& length \; (powset \; []) \\
& \qquad || powset \; 1 \\
& length \; [[]] \\
& \qquad || \\
& 1
\end{aligned}
\qquad \qquad \qquad
\begin{aligned}
& 2^{(length \; [])}\\
& \qquad || length 1 \\
& 2^0  \qquad \\
& \qquad || \\
& 1
\end{aligned}
\end{equation*}
%
\textbf{IV: } \\
IV 1: $length \; (powset \; xs) \equiv 2^{length \; xs}$ \\
IV 2: $length \; xs \equiv length \; [z|x <- xs]$ \\\\
%
\textbf{IS: } \\
%
\begin{align*}
& length \; (powset \; x:xs) \\
& \qquad || powset 1 \\
& length \; (powset' ++ [x:ys|ys <- powset']) \\
& \qquad || 2. Induktion \\
& (length \; powset') + (length \; [x:ys|ys <- powset']) \\
& \qquad || powset' \\
& (length \; xs) + (length [x:ys|ys <- xs]) \\
& \qquad || IV 2 \\
& (length \; xs) + (length \; xs) \\
& 2.(length \; xs) \\
& \qquad || IV 1 \\
& 2.2^{length \; xs} \\
& \qquad || 2.2^{n} = 2^{n+1} \\
& 2^{(length \; xs) + 1} \\
& \qquad || length 2 \\
& 2^{length \; (x:xs)}
\end{align*}
%
Zu zeigen: $length \; (xs ++ ys) \equiv (length \; xs) + (length \; ys)$ \\
Beweis per strukturellen Induktion \\
\textbf{IA: } \\
$length \; ([] ++ ys) \equiv length \; ys \equiv length \; ys + 0 \stackrel{length \; 1}{\equiv} legth []+ length \; ys$ \\
\textbf{IV:} \\
$length \; (xs ++ ys) \equiv (length \; xs) + (length \; ys)$ \\
\textbf{IS: } $xs -> x:xs$ \\
\begin{equation*}
\begin{aligned}
& length \; ((x:xs) ++ ys) \\
& \qquad || (:) \; Operator \\
& length \; ([x] ++ xs ++ ys) \\
& \qquad || (:) \; Operator \\
& length \; (x:(xs++ys)) \\
& \qquad  || length 2 \\
& 1 + length \; (xs ++ ys) \\
& \qquad || IV \\
& 1 + length \; xs + length \; ys \\
& \qquad || length 2 \\
& length \; (x:xs) + length \; ys
\end{aligned}
\end{equation*}

\section*{Aufgabe 3}
Zu zeigen: Leaves (Node lt rt) $\equiv$ SumLeaves (Node lt rt) \\
Beweis per strukturellen Induktion: \\ \\
\textbf{IA: } \\
sum Leaves (Leaf x) $\stackrel{?}{\equiv}$ 
	\text{(sumNodes lt) + 1}
\begin{equation*}
\begin{aligned}
& \qquad || \text{sumLeaves 1} \\
& 1
\end{aligned}
\qquad
\begin{aligned}
& \qquad ||\text{sumNodes 1} \\
& 0 + 1 \\
& \qquad || \\
& 1
\end{aligned}
\end{equation*}
\textbf{IV: } \\
sumLeaves lt $\equiv$ (sumNodes lt) + 1 \\
sumLeaves rt $\equiv$ (sumNodes rt) + 1 \\
\textbf{IS: }
\begin{align*}
& \text{sumLeaves (Node lt rt)} \\
& \qquad ||\text{sumLeaves 2} \\
& \text{(sumLeaves lt) + (sumLeaves rt)} \\
& \qquad || \text{IV. 1, IV. 2} \\
& \text{(sumNodes lt) + 1 + (sumNodes rt) + 1} \\
& \qquad || \text{sumNodes 2} \\
& \text{sumNodes (Node lt rt) + 1}
\end{align*}








\end{document}